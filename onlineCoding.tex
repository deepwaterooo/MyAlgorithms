% Created 2015-02-05 Thu 22:03
\documentclass[12pt]{book}
\usepackage{graphicx}
\usepackage{xcolor}
\usepackage{xeCJK}
\setCJKmainfont{SimSun}
\usepackage{longtable}
\usepackage{float}
\usepackage{textcomp}
\usepackage{geometry}
\geometry{left=1.5cm,right=1.5cm,top=2cm,bottom=1.5cm}
\usepackage{multirow}
\usepackage{multicol}
\usepackage{listings}
\usepackage{algorithm}
\usepackage{algorithmic}
\usepackage{latexsym}
\usepackage{natbib}
\usepackage{fancyhdr}
\usepackage[xetex,colorlinks=true,CJKbookmarks=true,linkcolor=blue,urlcolor=blue,menucolor=blue]{hyperref}


\lstset{language=Java,numbers=left,numberstyle=\tiny,basicstyle=\ttfamily\small,tabsize=4,frame=none,escapeinside=``,extendedchars=false,keywordstyle=\color{blue!70},commentstyle=\color{red!55!green!55!blue!55!},rulesepcolor=\color{red!20!green!20!blue!20!}}
\author{deepwaterooo}
\date{\today}
\title{Online Interview Questions - Problems Wise}
\hypersetup{
  pdfkeywords={},
  pdfsubject={},
  pdfcreator={Emacs 24.3.1 (Org mode 8.2.7c)}}
\begin{document}

\maketitle
\tableofcontents


\chapter{4 Sum, wrong answers}
\label{sec-1}

Given an array S of n integers, are there elements a, b, c, and d in S such that a + b + c + d = target? Find all unique quadruplets in the array which gives the sum of target.

Note

Elements in a quadruplet (a,b,c,d) must be in non-descending order. (ie, a ≤ b ≤ c ≤ d)

The solution set must not contain duplicate quadruplets.

Example

For example, given array S = \{1 0 -1 0 -2 2\}, and target = 0. A solution set is:
\lstset{language=java,label= ,caption= ,numbers=none}
\begin{lstlisting}
(-1, 0, 0, 1)
(-2, -1, 1, 2)
(-2, 0, 0, 2)
\end{lstlisting}
\lstset{language=java,label= ,caption= ,numbers=none}
\begin{lstlisting}
/**
 * @param numbers : Give an array numbersbers of n integer
 * @param target : you need to find four elements that's sum of target
 * @return : Find all unique quadruplets in the array which gives the sum of
 *           zero.
 */
public ArrayList<ArrayList<Integer>> fourSum(int[] numbers, int target) {     
    int n = numbers.length;
    HashSet<ArrayList<Integer>> set = new HashSet<ArrayList<Integer>>();
    ArrayList<ArrayList<Integer>> res = new ArrayList<ArrayList<Integer>>();
    if (n < 4) return res;
    Arrays.sort(numbers);
    int x = 0, y = 0, tmp = 0;
    for (int i = 0; i < n - 3; i++) {
        if (i > 0 && numbers[i] == numbers[i - 1]) continue;
        for (int j = i + 1; j < n - 2; j++) {
            x = j + 1;
            y = n - 1;
            while (x < y) {
                tmp = target - numbers[i] - numbers[j];
                if (numbers[x] + numbers[y] == tmp) {
                    set.add(new ArrayList<Integer>(Arrays.asList(numbers[i], numbers[j], numbers[x], numbers[y])));
                    x++;
                    y--;
                }
                else if (numbers[x] + numbers[y] < tmp) x++;
                else y--;
            }
        }
    }
    res.addAll(set);
    return res;
}
\end{lstlisting}
\chapter{2 Sum: O(1) Space, O(nlogn) Time undone}
\label{sec-2}

25\% Accepted

Given an array of integers, find two numbers such that they add up to a specific target number.

The function twoSum should return indices of the two numbers such that they add up to the target, where index1 must be less than index2. Please note that your returned answers (both index1 and index2) are not zero-based.

Note

You may assume that each input would have exactly one solution

Example

numbers=[2, 7, 11, 15], target=9

return [1, 2]

Challenge

\section{O(1) Space, O(nlogn) Time}
\label{sec-2-1}

\section{O(n) Space, O(n) Time}
\label{sec-2-2}
\lstset{language=java,label= ,caption= ,numbers=none}
\begin{lstlisting}
public int[] twoSum(int[] numbers, int target) { // O(nlogn), O(1)
    Arrays.sort(numbers);
    int [] res = new int[2];
    int i = 0, j = numbers.length - 1;
    while (i < j) {
        if (numbers[i] + numbers[j] == target) {
            res[0] = i + 1;
            res[1] = j + 1;
            return res;
        }
        if (numbers[i] + numbers[j] < target) i++;
        else j--;
    }
    return res;
}
\end{lstlisting}

\chapter{3 Sum}
\label{sec-3}

19\% Accepted

Given an array S of n integers, are there elements a, b, c in S such that a + b + c = 0? Find all unique triplets in the array which gives the sum of zero.

Note

Elements in a triplet (a,b,c) must be in non-descending order. (ie, a ≤ b ≤ c)

The solution set must not contain duplicate triplets.

Example

For example, given array S = \{-1 0 1 2 -1 -4\}, A solution set is:
\lstset{language=java,label= ,caption= ,numbers=none}
\begin{lstlisting}
(-1, 0, 1)
(-1, -1, 2)
\end{lstlisting}
\lstset{language=java,label= ,caption= ,numbers=none}
\begin{lstlisting}
public ArrayList<ArrayList<Integer>> threeSum(int[] numbers) {
    int n = numbers.length;
    ArrayList<ArrayList<Integer>> res = new ArrayList<ArrayList<Integer>>();
    HashSet<List<Integer>> set = new HashSet<List<Integer>>();
    Integer [] one = new Integer[3];
    int k = 0;
    for (int i = 0; i < n - 2; i++) {
        for (int j = i + 1; j < n - 1; j++) {
            k = j + 1;
            while (k < n - 1 && numbers[i] + numbers[j] + numbers[k] != 0) k++;
            if (k <= n - 1 && numbers[i] + numbers[j] + numbers[k] == 0) {
                one[0] = numbers[i];
                one[1] = numbers[j];
                one[2] = numbers[k];
                Arrays.sort(one);
                set.add(new ArrayList<Integer>(Arrays.asList(one)));
                k++;
            }
        }
    }
    for(List<Integer> i : set) 
        res.add(new ArrayList(i));
    return res;
}
\end{lstlisting}
\chapter{3 Sum Closest}
\label{sec-4}

30\% Accepted

Given an array S of n integers, find three integers in S such that the sum is closest to a given number, target. Return the sum of the three integers. 

Note

You may assume that each input would have exactly one solution.

Example

For example, given array S = \{-1 2 1 -4\}, and target = 1. The sum that is closest to the target is 2. (-1 + 2 + 1 = 2).

\lstset{language=java,label= ,caption= ,numbers=none}
\begin{lstlisting}
public int threeSumClosest(int[] numbers ,int target) {
    int n = numbers.length;
    int res = Integer.MAX_VALUE;
    int k = 0;
    for (int i = 0; i < n - 2; i++) {
        for (int j = i + 1; j < n - 1; j++) {
            k = j + 1;
            while (k <= n - 1) {
                if (Math.abs(numbers[i] + numbers[j] + numbers[k] - target) < Math.abs(res - target))
                    res = numbers[i] + numbers[j] + numbers[k];
                k++;
            }
        }
    }
    return res;
}
\end{lstlisting}
\chapter{A + B Problem, no idea;;;}
\label{sec-5}

60\% Accepted

For given numbers a and b in function aplusb, return the sum of them.

Note

You don't need to parse the input and output. Just calculate and return.

Example

If a = 1 and b = 2 return 3

Challenge

Can you do it without + operation?

Clarification

Are a and b both 32-bit integers?

\begin{itemize}
\item Yes.
\end{itemize}
\chapter{Anagrams My Submissions}
\label{sec-6}

28\% Accepted

Given an array of strings, return all groups of strings that are anagrams.

Note

All inputs will be in lower-case

Example
\lstset{language=java,label= ,caption= ,numbers=none}
\begin{lstlisting}
Given a string list: ["lint","intl","inlt","code"]
return ["lint","inlt","intl"]
\end{lstlisting}
\lstset{language=java,label= ,caption= ,numbers=none}
\begin{lstlisting}
public String mySort(String s) {
    char [] tmp = s.toCharArray();
    Arrays.sort(tmp);
    return new String(tmp);   //tmp.toString(); doesn't work
}

public List<String> anagrams(String[] strs) {
    Map<String, List<Integer>> m = new HashMap<String, List<Integer>>();
    ArrayList<String> res = new ArrayList<String>();
    for (int i = 0; i < strs.length; i++) {
        String tmp = mySort(strs[i]);
        if (!m.containsKey(tmp))
            m.put(tmp, new ArrayList<Integer>(Arrays.asList(i)));
        else m.get(tmp).add(i);
    }
    for (String key : m.keySet()) 
        if (m.get(key).size() > 1)
            for (int i = 0; i < m.get(key).size(); i++) 
                res.add(strs[m.get(key).get(i)]);
    return res;
}
\end{lstlisting}
\chapter{Backpack: still feeling difficult for me now\ldots{}}
\label{sec-7}

17\% Accepted

Given n items with size A[i], an integer m denotes the size of a backpack. How full you can fill this backpack? 

Note

You can not divide any item into small pieces.

Example

If we have 4 items with size [2, 3, 5, 7], the backpack size is 11, we can select 2, 3 and 5, so that the max size we can fill this backpack is 10. If the backpack size is 12. we can select [2, 3, 7] so that we can fulfill the backpack.

You function should return the max size we can fill in the given backpack.
\chapter{Balanced Binary Tree}
\label{sec-8}

46\% Accepted

Given a binary tree, determine if it is height-balanced.

For this problem, a height-balanced binary tree is defined as a binary tree in which the depth of the two subtrees of every node never differ by more than 1.

Example

Given binary tree A=\{3,9,20,\#,\#,15,7\}, B=\{3,\#,20,15,7\}
\lstset{language=java,label= ,caption= ,numbers=none}
\begin{lstlisting}
A)  3            B)    3 
   / \                  \
  9  20                 20
    /  \                / \
   15   7              15  7
\end{lstlisting}

The binary tree A is a height-balanced binary tree, but B is not.
% Emacs 24.3.1 (Org mode 8.2.7c)
\end{document}