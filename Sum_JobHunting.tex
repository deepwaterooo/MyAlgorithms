% Created 2015-02-01 Sun 13:18
\documentclass[12pt]{book}
\usepackage{graphicx}
\usepackage{xcolor}
\usepackage{xeCJK}
\setCJKmainfont{SimSun}
\usepackage{longtable}
\usepackage{float}
\usepackage{textcomp}
\usepackage{geometry}
\geometry{left=1.5cm,right=1.5cm,top=2cm,bottom=1.5cm}
\usepackage{multirow}
\usepackage{multicol}
\usepackage{listings}
\usepackage{algorithm}
\usepackage{algorithmic}
\usepackage{latexsym}
\usepackage{natbib}
\usepackage{fancyhdr}
\usepackage[xetex,colorlinks=true,CJKbookmarks=true,linkcolor=blue,urlcolor=blue,menucolor=blue]{hyperref}


\lstset{language=Java,numbers=left,numberstyle=\tiny,basicstyle=\ttfamily\small,tabsize=4,frame=none,escapeinside=``,extendedchars=false,keywordstyle=\color{blue!70},commentstyle=\color{red!55!green!55!blue!55!},rulesepcolor=\color{red!20!green!20!blue!20!}}
\author{deepwaterooo}
\date{\today}
\title{JobHunting One Year Interviewer Problems Summary}
\hypersetup{
  pdfkeywords={},
  pdfsubject={},
  pdfcreator={Emacs 24.3.1 (Org mode 8.2.7c)}}
\begin{document}

\maketitle
\tableofcontents


\chapter{数组}
\label{sec-1}
\begin{enumerate}
\item 一个sorted interger Array[1\ldots{}N], 已知范围 1\ldots{}N+1. 已知一个数字missing。 找该数字。 把原题改为unsorted,找missing数字。 performance。

\item 一个party N个人,如果一个人不认识任何其他人,又被任何其他人认识,此人为celebrity。用O(n)时间找到此celebrity。
\end{enumerate}

\chapter{字符串}
\label{sec-2}
\begin{enumerate}
\item 写atoi函数,测nodepad
\end{enumerate}

1.reverse words in a sentence,使用如下函数。
char* reverseWord(const char* str)

3.covert interger number to date string, for example, 20090130 -> "01/30/
2009"

第三道自己的做法是先取得30,然后01,然后2009然后组合成一个string。问题
是这样的话,月和日的01前的0可能会丢失,使得最后结果可能不对。后来想正
确的做法应该是先把整形转成string,然后使用substr并组合。另外,面试者问
有stl有什么可以替换itoa,自己答不出来。后来查了下,应该是可以使用
stringstream来实现,如下:
\lstset{language=java,label= ,caption= ,numbers=none}
\begin{lstlisting}
stringstream ss;
ss << intVal;
ss.str()就是我们要的结果。
\end{lstlisting}

\chapter{Linked List}
\label{sec-3}
\begin{enumerate}
\item 复制linked list。 已知每个节点有两个pointer,一个指向后一个节点,另一个指向其他任意一节点。 O(n)时间内,无附加内存,复制该linked list。(存储不连续)
\end{enumerate}

\chapter{Stack}
\label{sec-4}

\chapter{Tree}
\label{sec-5}
\begin{enumerate}
\item 给中序后续,构建树。
\end{enumerate}

\chapter{Hash Table}
\label{sec-6}

\chapter{Sort}
\label{sec-7}
2.an interger array containing millions of elements with min 0 and max 1000,
how to sort it?

第二道使用couting sort应该就可以。面试者要求描述算法,不需写代码。

\chapter{Search}
\label{sec-8}
. 一来就是比较高难度的,给你一个字节数组(注意取值范围),数组长度可能非常长,如何找到第一个只出现了一次的数字。开始没什么思路,和他讨论了一会,边问还边问复杂度和数据结构的问题,后来发现应该进行数出现次数,这样复杂度就是2n,结果出来了要求手写出代码。

\chapter{BFS}
\label{sec-9}

\chapter{DFS}
\label{sec-10}

\chapter{DC}
\label{sec-11}

\chapter{Greedy}
\label{sec-12}

\chapter{Backtracing}
\label{sec-13}
. 设计一个函数,返回一组数字的组合,combination,不同的是,你每调用它一次,它就返回下一个组合,而不是一次全返回。注:你不能一次算完,把他们存起来,而必须临时算!

\chapter{DP}
\label{sec-14}
\begin{enumerate}
\item 手写fab(n)函数,不是算,而是输出,递归或者循环都可,不过递归不高效大家应该知道
\end{enumerate}

\chapter{Two Pointers}
\label{sec-15}

\chapter{brain teaser}
\label{sec-16}
\begin{enumerate}
\item 50个黑球50个百球,2个罐,要求你放这100个球在这2个罐,使得别人随机从2个罐中任意拿一个球是黑球的几率达到最大。

\item heard on the street 上的男人出轨题,简单逻辑推理。

\item 2个人商量好策略,然后一个从52张牌里面随机抽5张,看牌,考虑。。。然后排在桌上,摊开前4张,第5张面朝下,由第二个人判断第5张牌。 问这个策略。
\end{enumerate}

\chapter{系统}
\label{sec-17}
\begin{enumerate}
\item 古老的三角形问题:输入3边,看是什么三角形。一个mobile device可以从服务器上传和下载图像,怎么测试这个系统?

\item lunch meeting之后回办公室打开电脑,说他们现在开发的某产品有问题,每次要loading很久,差不多10秒的样子。问怎么测试并找出这个bug? 这个把我难住了,胡乱讲了一通,然后说太困难了;于是他换了个题目,画了一个plotter软件的界面,问怎么测。
\end{enumerate}
coding的题目是Path Walk,给一条路径,写一个函数来走通它。其实这个题目我没搞明白什么意思,先沟通了很久,最后开始写(还是不太明白。汗\ldots{}),写完了觉得不正确,正想再改改,被打住了,说给个test case一起来看看程序怎么执行。每句代码跑了一通,却发现code写正确了:-) 

\begin{enumerate}
\item 一开始是个IQ题,把一堆数字填到格子里,满足一些条件,比如1和2不能相邻。测一个记事薄软件。有scheduler和notifier两部分,可以从scheduler输入时间和内容,然后notifier到预定时间会给出提醒。 coding题目很容易,找到单链表倒数第N个节点。
\end{enumerate}

系统设计和经验:
1 设计一个库,提供timer的功能。deltalist/hash,或类似linux kernal的 timer 设计。效率要比较高。
\begin{enumerate}
\item 一个类似chord的DHT设计。
\item 你有一个奇怪的程序,有时有bug,有时没有,说出尽可能多的可能原因。
\item printf来debug有何不妥。
\item process和thread。process之间的IPC有那些种?process间是否也可以sharememory.何时选thread或process。
\end{enumerate}
\url{http://www.mitbbs.com/article/JobHunting/31393101_3.html}

\chapter{c++}
\label{sec-18}
1.template中用typename和用class有什么区别?

2.unix下执行shell脚本和执行可执行文件有什么区别?哪个更快,为什么?脚本语言程序(如javascript)和可执行文件程序有什么区别?shell和这两者却别呢?

3.如何对const data member做assignment?
\lstset{language=java,label= ,caption= ,numbers=none}
\begin{lstlisting}
class A {
    const int a;
public:
    A():a(0){};
    A(int m_a):a(m_a){};
};

int main(){
    A a(1);
    A b;
    b = a; //how to implement assignment for this?
}
\end{lstlisting}

4.如果把base class对象赋给derived class对象,会怎么样?compiler报错还是执行错
误?
\lstset{language=java,label= ,caption= ,numbers=none}
\begin{lstlisting}
class A{
public:
    int a;
};

class B : public A{
public:
    int b;
};

int main(){
    A a;
    B b;
    b = a; //what happend? cout << b.b << endl;    
    B* b2;
    b2 = &a;  //how about this? cout << b->b << endl; 
}
\end{lstlisting}

\begin{enumerate}
\item 两个C的程序问题: 先是char*指针问题
\end{enumerate}
\lstset{language=java,label= ,caption= ,numbers=none}
\begin{lstlisting}
char *dosth() {
    char s[256];
    char * p = r;
    p = "some new string";
}
\end{lstlisting}

. 然后问了一堆变量的值,比如 s, *s, *(s+2), \&p, etc.

. 另外一个switch程序找错,没有加break之类,还有就是return local variable地址的
问题

\begin{enumerate}
\item 逻辑问题:八个水罐称重
\end{enumerate}

1, C vs C++

2, struct in C v.s. in C++ v.s. class in C++

3, virtual function, pure virtual function, abstract classwhat is the advantages of using virtual function

4, new v.s. malloc()

5, memory for a process (code, static data, stack, heap)

6, how to know the stack is growing in the direction of address increasing 
or decreasing

7, virtual memory

第一道被输入const给搞死了。先是没有注意const,直接按照常规非const做,没有写完就被叫停了;然后是被平时强调的malloc后必须及时delete规则搞死,坚持认为在函数里malloc一块内存然后在函数外delete是不好的习惯;最后当面试者提出如果定义一块内存,如char tmp\footnote{DEFINITION NOT FOUND.},然后使用会怎么样?自己提到可以在函数外strcpy函数返回结果,却忘了arr大小实际是无法指定的,所以这种方法是不可接受的。总之,很多的trick在里面没有注意到。

C++:effective c++上的东西若干;exception相关;继承和子父类指针若干. 十五分钟左右。
\begin{enumerate}
\item 大文件随机sample,one pass.
\item sodoku solver.
\item logn解x$^{\text{y}}$,
\item DP题
\item 1Billion query里选出时间最近5分钟内最frequent的1000个,one pass(我以前在amazon见到过这题)。
\end{enumerate}
6.两个排序数组找共同中值。递归和非递归解法。
7.斐波那契数列。100层楼梯下楼,可以一步也可以两步,多少种下法?递归和非递归。 
8 贝叶斯后验概率。
9。多少人在一起,生日可能出现重复概率大于0.5?(算法导论原题,我只记得个答案,直接说了。。。)
\begin{enumerate}
\item 一个数组,找最大值比较次数?同时找最大值和最小值比较次数?找最大值和次最大值比较次数?(他问我是否知道这题,我说是作业题。后来和师兄聊说是这他常拿来用的面试题。)
\end{enumerate}

\chapter{c++ and data structure}
\label{sec-19}
\section{single linked list, find nth from the end}
\label{sec-19-1}
\section{Overwriting and Overloading}
\label{sec-19-2}
\section{Stack vs. Queue}
\label{sec-19-3}
\section{Array of integers, all integers appear even times except one, find the one appears odd times. (some following up questions for this one)OOD}
\label{sec-19-4}
\section{Do you approve the following design?}
\label{sec-19-5}
\lstset{language=java,label= ,caption= ,numbers=none}
\begin{lstlisting}
Class Furniture{    
Some functions related to the property of furnitures;
};
\end{lstlisting}
4 classes derived from Furniture
\lstset{language=java,label= ,caption= ,numbers=none}
\begin{lstlisting}
Class wood_chair
Class steel_chair
Class wood_table
Class steel_table
\end{lstlisting}
What if you need to design a lot of other furnitures like desks\ldots{}. with other materials like plastics 
\section{An open question.Takes more time than any other questions. It is related to the project they are working on, you should not be asked.}
\label{sec-19-6}

\url{http://www.mitbbs.com/article/JobHunting/31487819_3.html}

\begin{enumerate}
\item given n strings with equal length, say x. find the substring shared by
\end{enumerate}
all of them. For example, abcx, abdx, abea, then ab is shared by all of them.

\begin{enumerate}
\item the gmail page loads very slow. any suggestion for improvement?

\item we want to check the number of querys obtained from the world in the last
\end{enumerate}
minute and the last hour, what data structure should you use for that? If 
there are billions of records, i.e, too many records for the main memory, 
what suggestions do you have?

马上就要第二轮店面了,求大家的题目和建议,我们也可以私下交流。谢谢!

\url{http://www.mitbbs.com/article/JobHunting/31487921_3.html}

你有一种语言的dictionary,你有一大串string,没有delimit,你如何interpret成字典中的字呢?

\url{http://www.mitbbs.com/article/JobHunting/31488093_3.html}

Given a binary tree
\lstset{language=java,label= ,caption= ,numbers=none}
\begin{lstlisting}
struct node {
    struct node* leftChild;
    struct node* rightChild;
    struct node* nextRight;
}
\end{lstlisting}
The nextRight points to the right node to the current node in the same level. Ask you populate the nextRight pointers in each node.

\url{http://www.mitbbs.com/article/JobHunting/31491521_3.html}

今天facebook第一面,现在hr都开始问技术问题了。。。问我会什么语言,我就说C++ best, 她就问我一些很基本C++的问题,还有两个bubble sort best case 的复杂度和一个排序的思路。 

面完之后给了puzzle的link,要求做meal和buffet里挑一个。 

\url{http://www.mitbbs.com/article/JobHunting/31494081_3.html}

1。Java里如何比较两个objects是否相等

2。怎样找出一个list是否包含循环 

3。inheritance和composition:什么时候需要用到哪种?

4。一个int array如何找出subarray,使得元素之和最大比如\{-2,3,-1,3,-4\}那么答案应该是\{3,-1,3\}

\url{http://www.mitbbs.com/article/JobHunting/31494489_3.html}

\chapter{OOD OOP}
\label{sec-20}
\begin{enumerate}
\item 一堆关于OO概念的问题,多态,继承,封装,接口和抽象类的区别,复写和重载(包括C++具体怎么实现的)
\end{enumerate}

然后就是一个智力问题,三个囚犯黑帽白帽,之前没见过,所以用了不少时间才想出来,大家可以搜搜,有现成的。 最后反问问题后结束。

\chapter{Google / MS}
\label{sec-21}

删除一个singly linked list节点, 但不知道head.不知道head, 怎么找之前的那个节点阿? 又没说有loop.谁知道trick在哪?

record the next node of it delete its next node do the assignment to copy saved next node to it 

\url{http://www.mitbbs.com/article/JobHunting/31454761_3.html}

Write code for finding number of zeros in n! OR Find the first
non-zero digit from the right in 100! (Factorial of hundred). Can an
int store hundred factorial. What size of array should be sufficient to solve the above problem. Write a code for the same.

\url{http://www.mitbbs.com/article/JobHunting/31454839_3.html}

you have a billion google searches a day, design a data structure
which lets you pull out the top 100 unique ones at the end of the
day.我的想法是create hashtable. scan billion data 一次,在hashtable纪
录每个query的次数, 然后再scan billion data一次,通过heap和hashtable找到
top 100, 不过这样的话,billion data会被scan 2次,disk i/o会很大
不知道有没有什么scan billion data一次就可以找到top 100的办法

\url{http://www.mitbbs.com/article/JobHunting/31455781_3.html}

\lstset{language=java,label= ,caption= ,numbers=none}
\begin{lstlisting}
list<int> L;
...
list<int>::iterator in_range =   find_if(L.begin(), L.end(),          compose2(logical_and<bool>(),                   bind2nd(greater_equal<int>(), 1),                   bind2nd(less_equal<int>(), 10)));
What is the best assertion that should be used as a post-condition?
assert(in_range == L.begin() || (*in_range >= 1 && *in_range <= 10));
assert(in_range == L.end() || (*in_range >= 1 || *in_range <= 10));
assert(*in_range >= 1 && *in_range <= 10);
assert(in_range == L.end() || (*in_range >= 1 && *in_range <= 10));
assert(in_range == L.end() && (*in_range >= 1 && *in_range <= 10));
\end{lstlisting}

\url{http://www.mitbbs.com/article/JobHunting/31456679_3.html}

he difference of following two expressions: Test A or Test B()
\lstset{language=java,label= ,caption= ,numbers=none}
\begin{lstlisting}
Class Test;
Test A;
or
Test B();
\end{lstlisting}

\url{http://www.mitbbs.com/article/JobHunting/31456597_3.html}
p
When a derived class is destructed, at what stage will the base class's 
destructor be called?
the answer varies on a case-by-case basis
It will automatically be called after the destructors for the derived class 
data members
It will automatically be called before the destructors for the derived class
data members
It should explicitly be called at the beginning of the derived class 
destructor
It should explicitly be called at the end of the derived class destructor
\url{http://www.mitbbs.com/article/JobHunting/31456485_3.html}

什么样的情况下用 virtual deconstructor?

\begin{enumerate}
\item virtual function是如何工作的?virtual table 是如何实现的?

\item virtual function具体调用哪个function是在编译的时候,还是在代码执行的时候决定的?

\item 类的copy constructor 和 assignment operator "=" 有什么区别?有什么主意事项?
\end{enumerate}

什么网站有C++的测试题库,哪有free的可以练练手?bloomberg 考的C++问题哪里能得到?

\url{http://www.mitbbs.com/article/JobHunting/31457805_3.html}

给定一个二叉树的一个node,编程返回中序遍历的下一个node。如果最后一个,返回null, 怎么做?

\url{http://www.mitbbs.com/article/JobHunting/31459733_3.html}

一堆数,其中一些数出现了一次,一些数出现了两次,只有一个数出现了三次找出那个出现了3次的数hash方法很trivial就不说了。如果用bitwise operator,怎么高效的做?除了XOR,是不是还得用点别的办法?

\url{http://www.mitbbs.com/article/JobHunting/31460327_3.html}

就是一直一篇文章,球可以覆盖所有单词的最小窗口,记得bbs有几个人提过这个题,但是没人给过解,希望牛人能够赐教!!感激!

\url{http://www.mitbbs.com/article_t/JobHunting/31460569.html}

\begin{enumerate}
\item If the probability of rain tomorrow is twice than no rain.What is the probability of rain tomorrow:

\item A grass, 3 cow can eat 3 days. 2 cow can eat 6 days. How long can one cowfinish the grass?
\end{enumerate}

\url{http://www.mitbbs.com/article/JobHunting/31461095_3.html}

Given a document and a query of K words, how do u find the smallest window 
that covers all the words at least once in that document? (given you know 
the inverted lists of all K words, that is, for each word, you have a list 
of all its occurrrences). This one is really hard. Could someone propose an 
algorithm in O(n)?

\url{http://www.mitbbs.com/article/JobHunting/31461767_3.html}

Desgin an algorithm to find whether a given sting is formed by the 
Intealeaving of two given strings. 注意,原来的两个given strings的本身的
character的顺序不能变。

这个题不简单,因为你不能简单的用3个指针分别指向三个string,遇到string A的就拷
贝到dst string,遇到string B的就拷贝他的。最麻烦的在于遇到A,B都相同的,你不能
advance both ptrs until they are different and then move one of them back. 
The point is who is to be moved back? You cannot simply randomly choose one.

For example, 

stringA: ABCEF\ldots{}

string B: ABCA\ldots{}

dst string : ABCABCEF\ldots{}.

那么,如果取B's ABCA 就错了。

\url{http://www.mitbbs.com/article/JobHunting/31463527_3.html}

Given a set of points (x,y) , find all pairs of points whose distance is less than a given number, say, K.这个题brute force: 对每个点,求和其他点距离,O(N$^{\text{2}}$),不知道哪位大侠有高见啊

\url{http://www.mitbbs.com/article/JobHunting/31463131_3.html}

\{1,5, -5, -8,2,  -1,15 \}要把负的扫到左边,正的扫到后边。不能改变顺序得到\{-5 -8 -1 1 5 2 15\}这个题有time 低于 n$^{\text{2}}$ space=O(1)的解法吗

\url{http://www.mitbbs.com/article/JobHunting/31464055_3.html}

这些东西我很都不熟悉。希望有高手指点指点,呵呵

\begin{enumerate}
\item Mempool design with 30k limit.mempool是应该在一开始就allocate 30k 连续的内存,然后分配和管理?或者是每次call allocate(n)的时候再通过operator new[]来分配内存,update size member?如果是的话,free(ptr, n)怎么写呢?貌似operator delete[]不能带size参数啊?总之我就是对memory design这块很不熟悉。。。

\item Implement put/get methods of a fixed size cache with LRU replacement algorithm.这个是不是用fixed size的max heap来实现?每个元素定义一个key,表示距离上次使用的时间,每使用一个元素,就相当于是把它的key更新为比当前最小值更小的数,然后做heapify()操作?每put一个元素,就assign新元素一个最小的key,然后用新元素替换掉堆顶点,然后做heapify?

\item Write a function to implement a buffer for DataOutputStream.这个我完全没啥概念了。。。求指点一下\textasciitilde{}
\end{enumerate}

4a. How do you write malloc and free to detect memory reference violation?

4b. flag a block of memory as used by putting some bit pattern at the beginning of the block. What bit pattern will you use?这两个问题也是摸不到头脑的。。。

\begin{enumerate}
\item How to implement singleton without using static/global variable?
\end{enumerate}

完全没思路,design pattern我基本上就是临时抱佛脚都还没抱上。。。

\url{http://www.mitbbs.com/article/JobHunting/31464509_3.html}

其实这些题也适合别的OS,只不过面试的这个职位是基于Linux的。

\begin{enumerate}
\item buffer overflow的工作原理:
\end{enumerate}

问的特定环境是: 在client-server的model下,client是如何通过网络造成server上的 buffer overflow,从而在server上制造出security hole?

俺只知道,肯定是client给server发的packet中,故意把特定的field(比如长度)弄错,使得server上的程序在copy的时候,造成buffer overflow (因为一个特大的length),谁能说说到底buffer overflow 是如何产生的?有什么好的文章,或者网站link介绍这个问题的?觉得搞network security的同学应该很明白。

\begin{enumerate}
\item 关于TCP的实现的问题(1): 操作系统中TCP的实现用到了几个timer,分别是什么?这个题怪怪的,谁能知道这样的细节?
\end{enumerate}

3.关于TCP的实现的问题(2): TCP packet header 中的Window size (接受方的 window size)的update是多久进行一次?就是问接收方在什么情况下,或者是多么频繁向发送方update新的windows size?

\begin{enumerate}
\item 关于socket: TCP connection 用socket建立后,有可能有很长时间通讯的双方没有任何数据来往,比如telnet client登录telnet server后,可能人会离开很长的时间,这个时候TCP server 怎么知道TCP client 是alive 还是crashed?如果你设计一个自己的应用程序,你该如何处理?就是问在你自己的client和server建立connection后,你是如何check whether the socket is still alive or not? 是在你自己的应用程序中定时的发一些类似于"Hello" 的packet作为查询呢?还是OS的socket能够自动的提供the information about the socket status?
\end{enumerate}

5.如何用C语言实现object oriented programming?

\begin{enumerate}
\item 关于kernel synchonization: 在SMP系统下,用spinlock,还是用semaphore来作synchronization比较好?为什么?俺只知道如果你的代码不能sleep的时候必须用spinlock,比如在interrup handler里面。还有就是如果用了spinlock,你要能够处理的很快。别的就想不出有什么区别了?不过Jonathan Corbet的"Linux device drivers"一书中说在Linux kernel 的实现中,spinlock引入的主要目的是为了让 Linux在SMP系统里运行的更有效,不知这是为什么?
\end{enumerate}

\url{http://www.mitbbs.com/article/JobHunting/31466547_3.html}

如题,职位是web engineer,希望有人可以用到。第一次电话是recruiter的,按清单问了些问题:
\begin{enumerate}
\item say some http methods?
\item get/put difference?
\item what does DTD for xml mean?
\item common protocol used in layer 4?
\item describe different ways to use css in html
\item difference between well-formed and valid xml?
\end{enumerate}

前两天第二轮technical phone interview:

\begin{enumerate}
\item why and how did u get into web development?
\item what do u like about web development? not like about it?
\item why do u want to work for google? 我扯到ajax的推广,他顺着问 ajax principle, security issue
\item what language are you comfortable with? talk about it. why and how did people design it?
\item explain 3 components of MVC
\item what happens when a user types google.com in URL bar and press enter?(dns, http get, tcp connection establishment, etc)
\item what may slow down the rendering of html page when its contents have been downloaded from server? (load other resources like css,js and parse them, etc)
\item read n lines of random numbers(space as delimiter) from a file, lines with same numbers are treated as duplicated lines, regardless of the order. check and print non-duplicate lines. performance time analysis.
\end{enumerate}

顺带问一个转身份的问题:如果我H1->F1->H1,重新转回H1的申请被拒了,那是不是还停留在正常的F1? 同样,其它转身份,例如 F1<->F2, F2<->H4互转之类,如果申请被拒,是正常停留在之前的身份吗? 前几天看到个帖子,H1 transfer被拒,身份就黑了

\url{http://www.mitbbs.com/article/JobHunting/31467259_3.html}

\begin{enumerate}
\item there are only 6 db connections in the pool, all 6 are being used,
\end{enumerate}
another request needs to connect to DB, it does not want to wait. How to do 
it? One solution is to make a new connection and add it to the pool. But the
interviewer wants standby solution. Anyone knows the standby solution? 
Thanks

\begin{enumerate}
\item For the db connection. min = 10 and max = 40.
\end{enumerate}
Will 10 connections be created at server start up time?

2.1 If we start with using 13 connections, when all the jobs are done, how 
many will be kept in the pool? 10 or 13?

2.2 If we start with using 45 connections, when all the jobs are done, how 
many will be kept in the pool? 40 or 45?

\url{http://www.mitbbs.com/article/JobHunting/31467451_3.html}

uppose there are n cities, and there may / may not be flight route
between c1 to c2. Design data structure to store this information and
write a function that receives two cities name, and return whether or
not there is a flight between them (either directly or through connections)

\url{http://www.mitbbs.com/article/JobHunting/31469019_3.html}

不trivial
Given a 3x3 square:
\lstset{language=java,label= ,caption= ,numbers=none}
\begin{lstlisting}
1 2 3
4 5 6
7 8 9
\end{lstlisting}
You are allowed to do circular shift on any row, and
circular shift on any column, as many times as you
please. Question: can you switch position of 1 and 2 with
the allowed circular shifts?

\url{http://www.mitbbs.com/article/JobHunting/31469459_3.html}

通常看到这种题目都感觉有点头疼。比如,design a messaging system. an online 
poker room.大家说说看

\url{http://www.mitbbs.com/article/JobHunting/31470087_3.html}

刚刚on-stie面试完某大公司。面试了7个人,大概问了20-30道题,有1道题不会,尽管
其他的都打上来了,很是郁闷,本以为自己准备的足够好了,哎。但是这道题不会,很
不甘心,希望大侠们帮助!!!

In our indexes, we have millions of URLs each of which has a link to the 
page content, now, suppose a user type a query with wild cards *, which 
represent 0 or multiple occcurrences of any characters, how to build the 
index such that such a type of query can be executed efficiently and the 
contents of all correpsonding URLs can be displayed to the users? For 
example, given a query \url{http://www.*o*ve*ou.com}. You man need to find iloveyou.com, itveabcu.com, etc. 
以前我见过类似用wild card来做query的,就*来说,一个方法是用*split
query into a few parts, for example, *o*ve*ou => o, ve, ou, 然后分别用o, ve, ou 查询,但是似乎不适合这道题。 另外,如果对Index里的每一个URL建suffix tree ,然后对每个query check againgt 所有的suffix tree, 这样实际上就是scan all urls, 明显也不合适。但是排序?我想不出来。

\url{http://www.mitbbs.com/article/JobHunting/31472965_3.html}

\chapter{Google interview question}
\label{sec-22}
Design a system to store heap on multiple machines ? What is avg number of 
machines accessed per operation and  number of elements stored in a machine ?
First greater number in an array. Given a large array of positive integers, 
for an arbitrary integer A, we want to know the first integer in the array 
which is greater than or equal A . O(logn) solution required
\lstset{language=java,label= ,caption= ,numbers=none}
\begin{lstlisting}
ex  [2, 10,5,6,80]
input : 6     output : 10
input :20    output : 80
\end{lstlisting}

Given an N-by-N array of black (1) and white (0) pixels, find the largest 
contiguous sub-array that consists of entirely black pixels. In the example 
below there is a 6-by-2 sub-array.

\lstset{language=java,label= ,caption= ,numbers=none}
\begin{lstlisting}
1 0 1 1 1 0 0 0
0 0 0 1 0 1 0 0
0 0 1 1 1 0 0 0
0 0 1 1 1 0 1 0
0 0 1 1 1 1 1 1
0 1 0 1 1 1 1 0
0 1 0 1 1 1 1 0
0 0 0 1 1 1 1 0
\end{lstlisting}

\url{http://www.mitbbs.com/article/JobHunting/31487235_3.html}

Given a log file, which contains a series of websites, which the user has 
visited, find the most frequent path of 3 websites.

e.g: If this is a log file
\lstset{language=java,label= ,caption= ,numbers=none}
\begin{lstlisting}
A B C D E
A C D B E
C D E B A
A C D E B
C D E A B
\end{lstlisting}

clearly, C D E in the most frequent website?

\url{http://www.mitbbs.com/article/JobHunting/31493409_3.html}

\begin{enumerate}
\item find a pair that add up to a given sum

\item find all phone numbers in the html pages in a folder (and subfolder). something else, and self-introduction stuff
\end{enumerate}

\url{http://www.mitbbs.com/article/JobHunting/31493961_3.html}
\chapter{google interview question from glassdoor}
\label{sec-23}
Design and describe a system/application that will most efficiently produce 
a report of the top 1 million Google search requests. You are given:

You are given 12 servers to work with. They are all dual-processor machines 
with 4Gb of RAM, 4x400GB hard drives and networked together.(Basically, 
nothing more than high-end PC's)

The log data has already been cleaned for you. It consists of 100 Billion 
log lines, broken down into 12 320 GB files of 40-byte search terms per line.
You can use only custom written applications or available free open-source 
software.

\url{http://www.mitbbs.com/article/JobHunting/31483445_3.html}

u are given a binary search tree,
each node has a parent, left and right
do pre-order/in-order traversal without stack.
cannot change the structure of Node.
test cases: 8  6  7  5  4  9  10  11  12
test your codes using the test case above.
\url{http://www.mitbbs.com/article/JobHunting/31483789_3.html}

关于排列组合的程序问题, 我一只都没理解太清楚, 现在厚脸皮来请教一下. 这些问题一般都要涉及到递归, 我这里不是问的算法的问题, 而是程序的实现问题. 我一直不知道怎么实现才是对的. 比如, 5 选 3 的全组合, a,b,c,d,e. 

1 中间结果怎么保存, 是用一个vector来保存,还是用多个vector来保存?

2 如果用一个vector来保存, 递归的时候, 最终状态是什么? 何时pop, 何时push, ?

\url{http://www.mitbbs.com/article/JobHunting/31484637_3.html}

Given an array, find the longest subarray which the sum of the
subarray less or equal then the given MaxSum.
\lstset{language=java,label= ,caption= ,numbers=none}
\begin{lstlisting}
int[] FindMaxSumArray(int[] array, int maxsum)
for example, given array: {1, -2, 4, 5, -2, 6, 7}
maxsum=7
the result would be: {1,-2, 4, -2, 6}
\end{lstlisting}

\url{http://www.mitbbs.com/article/JobHunting/31484653_3.html}

given a integer, output its previous and next neighbor number which
has the same number of bit 1 in their binary representation.下面为什么
去判断(number \& 3) != 2?
\lstset{language=java,label= ,caption= ,numbers=none}
\begin{lstlisting}
while ((number & 3) != 2) { // for right neighbor, change this line to 
// (number & 3) != 1
\end{lstlisting}

\url{http://www.mitbbs.com/article/JobHunting/31485091_3.html}

要求当场写code。 

1  下面的int * takeaddress()有没有问题, 啥问题? 

2  写个效率高的takeaddress出来, 实现同样的功能
\lstset{language=java,label= ,caption= ,numbers=none}
\begin{lstlisting}
int  * paddress, address1, *r;
paddress = takeaddress(); /* defined below */
address1= paddress[0];

int * takeaddress()
{int  address[8];
/* The address are defined here */
  return  address;
}
\end{lstlisting}
\url{http://www.mitbbs.com/article/JobHunting/31485465_3.html}

一个字符串,要求返回重复次数最多且最长的子字符串(假设源字符串中最长重
复次数最多的子字符串只有一个)。例如 “abcabcdfabcdf”要求返回
“abcdf”. 因为“abcdf”重复次数最多且最长。俺只想到两个土办法:

1)找到所有字符串组合(例如a, ab, abc, abca, b, bc, \ldots{}.),都放入hash table,找重复次数最多的且最长的。

2)用Dynamic Programming找LCS的办法,两个字符串都是源字符串,然后在那个2D array里面找最长match,并计算重复的次数,然后输出结果。

感觉两个方法的time complexity都挺大的,不知大家有没有什么别的想法?

\url{http://www.mitbbs.com/article/JobHunting/31485529_3.html}

很多都是老题,不过我专门整理了一下:

\begin{enumerate}
\item string match: string Text, Pattern; find a substring of Text matches with Pattern.
\end{enumerate}

解法纲要:Rabin-Karp, KMP, suffix tree

变种1b: multiple match: string Text, PatternSet[n]; find a substring of Text matches with any one pattern in the set;

解法纲要: Rabin-Karp

2.LCSubstring: string A,B; find the longest common consecutive substring;

解法纲要:DP(A.len*B.len复杂度),suffix tree(A.len+B.len复杂度) 

3.Longest Palindrome: string A; find the longest substring of A which is a palindrome;

解法纲要:类似2

4.Wild card match:

4a: Pattern contains '?'(s)

4b: Pattern contains '*'(s)

4c: Pattern contains both;

//以下是与dictionary有关的题目
\begin{enumerate}
\item dictionary + wild card search(一般都需要做适当预处理):
\end{enumerate}

第一种search:search所有match结果

第二种:返回某个特定的结果,比如,所有match中最长的单词

5a: pattern = ??a????b* (指定某些位上的字母)

5b: pattern = abcde* (指定fixed/unfixed length的前缀)

5c: pattern = ?a*bcd*e?f* (?和*任意混合搜索)

解法:待探讨

\begin{enumerate}
\item dictionary + 包含字符集合:
\end{enumerate}

Letter$_{\text{Set}}$ = "aabbbcd";

第一种search: 所有至少包含2个a,3个b,1个c,1个d的单词

第二种search:所有至少包含这个字母集合的单词中最长的/最短的

解法:待探讨

\begin{enumerate}
\item convert a valid word to another valid word of the same length, by replacing one letter in one step, every intermediate word must also be valid;
\end{enumerate}

解题思路:相同长度的单词构建一个图 + BFS

\begin{enumerate}
\item edit distance (misspell correction): type a misppell word, give top10/all suggestions of correct words;
\end{enumerate}

解题思路:首先定义计算edit distance的metrics,然后从每个valid单词计算出到它距离<=某给定值的所有misspell的单词(类似BFS的一层一层的算)

\begin{enumerate}
\item find a matrix with max area: each row and each column of the matrix must be a valid word;

\item 朴素搜索,在dictionary中搜索一个单词是否存在:
\end{enumerate}

解题思路:hash; trie; 

10b. shortest unique prefix: give a string, find its shortest prefix, which doesn't match with any prefix of any valid word in dictionary;

for example:

cat against \{dog, be, cut\} is ca

cat against \{dog, be, cut, car\} is cat

cat against \{dog, be, cut, car, cat\} is null 

解题思路:trie/prefix tree;

\begin{enumerate}
\item solve a crossword puzzle;
\end{enumerate}

\url{http://www.mitbbs.com/article/JobHunting/31485923_3.html}

\begin{enumerate}
\item N台机器,每台机器有N个数找median (2个数组找median的扩展版)

\item 已知coin denominator set,例如,2cent, 3cent, 5cent\ldots{}给定一个目标数,比如126centsk这个题我以前问过一次,没人回。。。我觉得是很好的题,贪心,回溯,DP都可以试试。但是我一直没找到最满意的解。

\item 一个整数数组,找3个数满足勾股定理。求比O(n$^{\text{2}}$)更好的解
\end{enumerate}

\url{http://www.mitbbs.com/article/JobHunting/31486805_3.html}

glassdoor上看到一道题目:

Given a file of unknown size, devise an algorithm to give equal probability randomization to choosing a single line given a one line buffer space.

\url{http://www.mitbbs.com/article/JobHunting/31487119_3.html}

\begin{enumerate}
\item 很长的log file记录了用户访问amazon.com的过程,两列分别为 userID 和 pageName.
\end{enumerate}

log从上倒下按照点击发生的时间顺序。找出最popular的3连击。
\lstset{language=java,label= ,caption= ,numbers=none}
\begin{lstlisting}
zhang  welcome
Li     Hello
Wang   welcome
Li     books
Wang   Hello
zhang  books
Li     shopping cart
Li     checkout
zhang  shopping cart
Wang   camera
zhang  checkout
\end{lstlisting}

最popular的3 combo是books -> shopping cart -> checkout

\begin{enumerate}
\item Permutation of a string.这题最郁闷,我把programming expose里的code默写了出来。但这个方法是不管字符重复的,假设都是不同的。现在考官要不显示重复的,而且他要求不能先都列出来再剔除,而要在发现重复的时候及时制止。没想出来

\item Design a fight ticket booking system.

\item 老板说网站很慢怎么办?老板说数据库很慢怎么办?
\end{enumerate}

\url{http://www.mitbbs.com/article/JobHunting/31487345_3.html}

\chapter{LinkedIn}
\label{sec-24}
\begin{enumerate}
\item 2D matrix, sorted on each row, first element of next row is larger(or
\end{enumerate}
equal) to the last element of previous row, now giving a target number, 
returning the position that the target locates within the matrix

\begin{enumerate}
\item Given a binary tree where all the right nodes are leaf nodes, flip it
\end{enumerate}
upside down and turn it into a tree with left leaf nodes.

for example, turn these:
\lstset{language=java,label= ,caption= ,numbers=none}
\begin{lstlisting}
      1                1
     /               / 
    2   3            2   3
   / 
  4   5
 / 
6   7
\end{lstlisting}

into these:
\lstset{language=java,label= ,caption= ,numbers=none}
\begin{lstlisting}
      1               1
     /               /
    2---3           2---3
   /
  4---5
 /
6---7
\end{lstlisting}

where 6 is the new root node for the left tree, and 2 for the right tree.

oriented correctly:
\lstset{language=java,label= ,caption= ,numbers=none}
\begin{lstlisting}
  6                  2
 /                 / 
7   4              3   1
   / 
  5   2
     / 
    3   1
\end{lstlisting}

\begin{enumerate}
\item 电面不用Gdoc,用CollabEdit

\item 第一题其实是LC原题的变种,等于的边界情况稍微处理一下就可以了
\end{enumerate}

\url{http://www.mitbbs.com/article_t1/JobHunting/32775405_0_1.html}

\begin{enumerate}
\item 层序打印 binary tree

\item 实现 BlockingQueue 的 take() 和 put()
\end{enumerate}
\lstset{language=java,label= ,caption= ,numbers=none}
\begin{lstlisting}
public interface BlockingQueue<T>
{
    /** Retrieve and remove the head of the queue, waiting if no elements 
are present. */
    T take();

    /** Add the given element to the end of the queue, waiting if necessary 
for space to become available. */
    void put (T obj);
}
\end{lstlisting}

\begin{enumerate}
\item 实现一共 TwoSum interface
\end{enumerate}
\lstset{language=java,label= ,caption= ,numbers=none}
\begin{lstlisting}
public interface TwoSum {
    /**
     * Stores @param input in an internal data structure.
     */
    void store(int input);

    /**
     * Returns true if there is any pair of numbers in the internal data 
structure which
     * have sum @param val, and false otherwise.
     * For example, if the numbers 1, -2, 3, and 6 had been stored,
     * the method should return true for 4, -1, and 9, but false for 10, 5, 
and 0
     */
    boolean test(int val);
}
\end{lstlisting}
\url{http://www.mitbbs.com/article_t/JobHunting/32802467.html}

\chapter{Amazon}
\label{sec-25}
\section{为什么对Amazon感兴趣。}
\label{sec-25-1}
\section{自己最近的Project。}
\label{sec-25-2}
\section{说出自己会的编程语言并打分(1-5)。}
\label{sec-25-3}
\section{有没有开发Mobile application的经验。}
\label{sec-25-4}
\section{几个常见Data structure的Lookup操作的时间复杂度。}
\label{sec-25-5}
\section{HTTP post和get的区别。}
\label{sec-25-6}
\section{Design Pattern: Singleton, Factory, Lazy initialization。}
\label{sec-25-7}
\section{Multi-threaded programming, deadlock之类。}
\label{sec-25-8}
\section{对Unix环境是否熟悉,几个常见命令,ls, ps之类。}
\label{sec-25-9}
\section{Reflection的概念,Java reflection,C++里面是不是有reflection。}
\label{sec-25-10}
\section{如何实现Garbage Collection。Reference counting的缺点(cycle),如何解决,JVM有没有解决。}
\label{sec-25-11}
\section{C++里面virtual destructor的用途,于一般virtual function的区别。}
\label{sec-25-12}
\section{写一个函数实现两个整数相除,不用"/"和"\%",返回商和余数。写完读给他听。}
\label{sec-25-13}
\section{算法设计:一个Galaxy,每个星星用一个三围座标表示,找出离地球最近的1000个。}
\label{sec-25-14}

\chapter{amazon}
\label{sec-26}
\section{那道wood steel table chair furniture的题目}
\label{sec-26-1}
\lstset{language=java,label= ,caption= ,numbers=none}
\begin{lstlisting}
#include <iostream>
using namespace std;

class stuff {
public:
    stuff() {}
    virtual ~stuff() {}
    virtual void info() = 0 ;
};

class table : public stuff {
public:
    table() {}
    ~table() {}
    void info() {
        cout << "Table " << endl;
    }
};

class chair : public stuff {
public:
    chair(){}
    ~chair(){}
    void info() {
        cout << "Chair" << endl;
    }
};

class wood: public  stuff {
public:
    wood(stuff * s): stf(s) { }
    void info() {
        cout << "Wood ";
        stf->info();
    }
private:
    stuff* stf;
} ;

class steel : public stuff {
public:
    steel(stuff *s) : stf(s) {}
    void info() {
        cout << "Steel ";
        stf->info();
    }
private:
    stuff* stf;
};

int main() {  stuff * wood_chair = new wood(new chair); 
    stuff * wood_table = new wood(new table); 
    stuff * steel_chair = new steel(new chair); 
    stuff * steel_table = new steel(new table); 
    stuff * wood_steel_chair = new wood(new steel(new chair)); 
    wood_chair->info(); 
    wood_table->info(); 
    steel_chair->info(); 
    steel_table->info(); 
    wood_steel_chair->info(); 
    delete wood_chair; 
    delete wood_table; 
    delete steel_chair; 
    delete steel_stable; 
    delete wood_steel_chair; 
}
\end{lstlisting}

\lstset{language=java,label= ,caption= ,numbers=none}
\begin{lstlisting}
output:
Wood Chair
Wood Table 
Steel Chair
Steel Table 
Wood Steel Chair
\end{lstlisting}

如果需要plastic, 只需要再从stuff inherit一个plastic类就行了

\texttt{=========================另一种方案==================================}
\lstset{language=java,label= ,caption= ,numbers=none}
\begin{lstlisting}
#include <iostream>
using namespace std;

class material {
public:
    material() {}
    virtual ~material() {}
    virtual void info() = 0 ; 
};

class wood : public material{
public:
    void info() {
        cout << "Wood ";
    }
};

class steel: public material {
public:
    void info() {
        cout << "Steel ";
    }
};

class furniture {
public:
    furniture() {};
    void setMaterial(material *m) {
        this->m = m;
    }
    virtual ~furniture() {};
    virtual void info() = 0 ;
protected:
    material * m;
};

class table : public furniture {
public:
    table() {};
    void info() {
        m->info();
        cout << " Table" << endl;
    }
};

class chair : public furniture {
public:
    chair() {};
    void info() {
        m->info();
        cout << " Chair" << endl;
    }
};

int main() {
    table *wood_table = new table();
    wood_table->setMaterial(new wood());
    chair *steel_chair = new chair();
    steel_chair->setMaterial(new steel());
    wood_table->info();
    steel_chair->info();
    delete wood_table;
    delete steel_chair;
}
\end{lstlisting}

output is

\lstset{language=java,label= ,caption= ,numbers=none}
\begin{lstlisting}
Wood  Table
Steel  Chair
\end{lstlisting}

\url{http://www.mitbbs.com/article/JobHunting/31494857_3.html}

uppose that you have a set of nodes with no null pointers (each node points
to itself or to some other node in the set), given a pointer to a node, how
to find the number of different nodes that it ultimately researches by 
following links from that node, without modifying any nodes. DO NOT use more
than a constant amount of extra memory spa

\url{http://www.mitbbs.com/article/JobHunting/31495985_3.html}

\begin{enumerate}
\item 给定一个首尾相连的排过序的单链表,首节点最大尾节点最小,给出链表中任意一个节点,要求返回链表中间节点;

\item 一摞未排序的扑克中间有重复,用最有效的方法找出并删除重复者
\end{enumerate}

\url{http://www.mitbbs.com/article/JobHunting/31496467_3.html}

昨天去某公司面试 Software Engineer碰到的最后一道题:

有一种新语言,只能做三种操作。

X=0;  给变量赋值为0;

X++;  递增

LOOP(x)\{。。\}   给定一个变量值就循环X次,循环block可以嵌套定义的三种操作。

题目是给定B,求A=B-1。

\url{http://www.mitbbs.com/article/JobHunting/31496897_3.html}

\begin{enumerate}
\item online skill assessment, Dec 2009 Some like GRE questions.
\item phone interview with 2 people in R\&D, Dec 2009
\item how to find 1 missing number from 0 to N in an array of N numbers.
\item brainteaser, 5 jar problems.
\item how to calculate sqrt(N) without using sqrt function. Binary search tree problem.
\item some behavioral problem. Like, How do you know about BB? Why you wanna work in BB? Why you wanna work in industry?
\item onsite interview, Jan 2010 1st meet 2 people in R\&D
\item train, tunnel, people escaping problem
\item 6 digits number, each changes from 0 to 9. Find the odds that sum
of first three is the same as the sum of last three. A: 2 do loop.
\item Find 1 missing number from 0 to N. But notice that it is possible
the sum would overflow. Think about a way to avoid the overflow.
\item Tricky problem. I do not think anyone else would know the answer except the one who gives the problem. Nothing to do with math, statistics.
\item Same 5 jars problem. That is their favorite.
\end{enumerate}

2nd meet a lady in HR.

Ask 15-20 Behavioral problems. Cover most commonly behavioral problems.

3rd meet a senior manager in R\&D

Ask one question, how to find the first unique number in an array of byte. and write a code to realize it.

\url{http://www.mitbbs.com/article/JobHunting/31497519_3.html}

Suppose there is a C function to count and return thhe number of nodes in a linked list.

What cases would you cover in unit tests of this function?

I can only think of two testing cases

(1): An empty list.

(2): An extrem long list with the length of the maximum value of unsigned int.

\url{http://www.mitbbs.com/article/JobHunting/31499799_3.html}

1.behavior question, Why you want to join BB?

2.一个windows系统,一个unix系统,unix系统里有100个数据库,总共1TB,如何在1小时内从unix系统转移到windows系统中
3.找出一个字符串中最早出现的非重复字母
4.两个鸡蛋测试那层楼丢下来会碎
5.问了些做过的project的具体内容另,我想再联系下我的面试官,我知道名字,怎么找到他的邮箱地址?谢谢。

\url{http://www.mitbbs.com/article/JobHunting/31499929_3.html}

有一个循环链表 a->d->b->c->e->\ldots{}.->a, 每一个节点都是一个整数,且不重复(除了首尾节点外)。现在这个链表被拆断开了,每2个相邻节点被存在一个cell里面, 但这些cell不是有序的。 就是说链表被拆成了 a->d, c->e,\ldots{},d->b,\ldots{},b->c,\ldots{}. 我想重新把链表建立起来,应该用什么样的算法?

\url{http://www.mitbbs.com/article/JobHunting/31500287_3.html}

合并两个BST要求O(n+m)时间,n和m为两棵树的大小。有什么好的解法么?

\url{http://www.mitbbs.com/article/JobHunting/31500627_3.html}

面试了一个小时左右。

\begin{enumerate}
\item 用两个stacks来实现一个queue,题不是很难,但是要求逐行念代码,精确到冒号分号,尖括号怎么说不知道。。。耗了好久。
\item 一些关于multi-threading,critical section,等等。
\item SQL的一些问题,我不怎么会,就skip了。
\item OOD问题,如何设计parking garage,大家有什么好的想法吗?
\end{enumerate}

\url{http://www.mitbbs.com/article/JobHunting/31501235_3.html}

给你一个字典array of strings (you may preprocess it if necessary)任意一个单词,求最小的edit distance一个单位的distance定义为:

a. replace a letter

b. delete a letter

c. insert a letter (also at any position)

快速的code出来~ 你就可以拿facebook面试了

\url{http://www.mitbbs.com/article/JobHunting/31501445_3.html}

\section{题目}
\label{sec-26-2}

题目1. LIS. 一个任意的数组,找出一个严格单调递增的最长子序列。例如: \{3,0,1,7,2,4,5,9\} –> output: \{0, 1, 2, 4, 5, 9\}很简洁巧妙的算法,能在O(N log N)时间和O(N)空间做出来!方法就是始终保持一个单增的序列,然后新来的数如果比当前最大还大就append在后面,否则在单增序列里面做binary search,替换相应位置的数。

题目2. 玻璃杯/鸡蛋drop问题。有N层楼,假定是在 i 层楼扔鸡蛋,如果没有碎,那么在所有<=i 楼层扔鸡蛋都保证不会碎,反之如果碎了,那么保证在所有 >=i 楼层扔鸡蛋都必碎。通过若干次尝试扔鸡蛋,找到某个鸡蛋碎/不碎的”临界”层。允许你扔鸡蛋的总次数是D,允许你打碎的鸡蛋数是B。

问题的描述是:对一组给定的数(N D B),如果存在一个策略保证能在D B的限制下,在N层楼中找到“临界”层,那么称此(N D B)是Solvable的。接下来相关联的三个问题就是:

(a)给定D,B,求满足(N,D,B)Solvable的最大的N$_{\text{max}}$. 例:D=4,B=1, 策略是从第一层开始一层层往上. N$_{\text{max}}$=D=4.

(b)给定F,B,求最小的D$_{\text{min}}$

(c)给定F,D,求最小的B$_{\text{min}}$

这个问题相当容易找到看似最优的解,但是绝大部分的方法都不是最优的(最快最高效)。而且最迷惑人的是,(a)(b)(c)三个问题中,必须先从其中某一个下手开始解决,如果你不幸的先从另外的两个问题下手,多半离最优解遥遥无望。

如果你找到了正确的入手点,有了正确的思路,最后的答案会异常的简单!入手点就是首先解决(a)问题,并且可以递归的来解决:假设D,B对应的答案是F(D,B),那么考虑在某一层摔一个鸡蛋后,如果碎了,D--,B--,如果没碎就只是D--,B不变。这样很容易写出递归方程,算出F关于D,B的table。

题目3. 经典的概率悖论。3扇门,一扇背后有羊,你选中一扇门后,现在另外一扇门开了,里面是空的。问你是否应该重新选择。

分析:据观察,有一部分的人坚持认为一定要重新选择,另一部分的人认为是否重新选择都一样。另外少部分的人能看出,这个问题很巧妙的隐含了意识(主观intention),信息和概率的关系!

题目4. 很简单的,N个数的数组,找出最大的和第二大的数,只用N+logN-2的比较次数,不需要额外空间。这个是典型的问题本身就是答案提示的题目--基于比较又有LogN,很显然思路涉及二分法,继续下去,剩下的问题就仅仅是找一个符合要求的Implementation了。

题目5. 找N!最后一个非零的数字。巧妙的方法可以在 LogN 时间内找出来,一个hint是利用 5$^{\text{k(和log}}_{\text{5)划分问题}}$

题目6. 任务分配,假设有N个任务,每个任务需要W$_{\text{i工作量,M个人,每人每天能做工作量w}}$$_{\text{i,如何安排工作,使得所有工作能最快完成。这个问题其实更像一个开放性问题,因为一个合理的贪心策略,最后的结果跟最优结是很接近的}}$(大致上,最多只差一天)。

题目7. 计算Fibonacci 数 F(n),O(n)的算法是很trival的。但是有很漂亮简洁的Log(N)算法,思路是利用2*2矩阵表示Fibonacci递推式,然后用二分法的思想球矩阵的N次方。

题目8. 一颗BinaryTree,每个节点有个NULL指针,要求把每个节点和在BFS中它
的下一个节点串起来。其他BinaryTree的常见题有比如非递归的实现遍历,
用.parent or stack。思考这些题的经验是,对于这一类的树的题目,有很强的
递归性/规律性,通常都是O(N)的复杂度,那么把N steps的问题,放在某个单
step来研究,会把思路变得更清晰。另外一点就是,完全可以假设在做这一单步
之前,在做这一步之前的问题已经最大可能的正确解决了,这样能够以一种数学
归纳法的思想去利用之前的结论。比如这个题里面,假设节点 i 之前的节点都
已经串好了,如何把 i 串到下一个节点。这个问题就是看一眼草图就能知道的
了。最后一点经验是,在效率相当的算法的基础上,不同版本的实现,已经有能
够互相启发的地方。

\url{http://www.mitbbs.com/article/JobHunting/31502251_3.html}

第一题。给一个数组a\footnote{DEFINITION NOT FOUND.}到a[n] : 例如 1,2,3,4,5,6. 现在随机生成a的
一个permutation: b\footnotemark[2]{}到b[n] (例如:3 1 5 2 4 6)问, a和b数组在每一位
上都不相同的概率是多少?假设a本身没有重复的数 

主问题:F(n) = 给定长度为n的a数组,b数组有多少种取法辅助问题:结果用f(n)表示。 b数组是\{1….i-1,x,i+1…n\}的一个排列,其中x!=i,满足a,b在每一位上都不相同,有多少种b?例如,a = 1,2,3,4; b是\{1,2,5,4\}的一个排列。换句话说,组成b的元素中,有且只有一个数不在a中。这样定义了F(n),f(n)后,很显然有递推关系:

F(n) = (n-1) * f(n-1)    //解释:第一位有n-1种选择,任意一种选择后,问题变为一个 n-1规模的辅助问题

f(n) = F(n-1) + (n-1)*f(n-1)   //情况一,在b数组的第i位置填入x,考虑剩下的n-1个位置,即是一个n-1规模的主问题;情况二,i位置填入非x的数,考虑剩下的n-1个位置,即是一个n-1规模的辅助问题。

简化一下表达式就是:

F(n) = (n-1)(F(n-1)+F(n-2))

第二题,一个binary tree,逆序打印BFS序列。不能同时用两段存储空间(不同时用queue和stack)

解法,用一个vector(array)模拟queue+stack。queue的push操作即vector的push$_{\text{back,等效于}}$ q.pop()+stack.push()的操作则是,vector的index往前走一步!最后把vector从尾到头打印一遍即可。

第三题,网上看的答案,超级巧妙,生成一个0-255 二进制数有多少位是1的查询表
\lstset{language=java,label= ,caption= ,numbers=none}
\begin{lstlisting}
static int BitSetCount256[256] = {
#define B2(n) n, n+1, n+1, n+2,
#define B4(n) B2(n), B2(n+1), B2(n+1), B2(n+2),
#define B6(n) B4(n), B4(n+1), B4(n+1), B4(n+2),
    B6(0), B6(1), B6(1), B6(2)
}
\end{lstlisting}

不得不说,这个宏递归的方法用的太妙了!!!附带赞一个巧妙度略低一些的计
算二进制数有多少位1的方法
\lstset{language=java,label= ,caption= ,numbers=none}
\begin{lstlisting}
int bitSetCount(unsigned int i){
    int c=0;
    while (i) {
        c++;
        i &= (i-1);  //这一步很赞,每次保证清除最低一位1;
    }
    return c;
}
\end{lstlisting}

\url{http://www.mitbbs.com/article/JobHunting/31502237_3.html}

\section{学习了backtrack(回溯法)}
\label{sec-26-3}

之前做了一些回溯的题,比如打印permutation,打印任意n对括号等等,都是瞎蒙的。还真凑巧,上午做了打印n括号的题,下午就看见有人说到回溯法,想想自己还没系统学过这个,找了本基础的中文算法书来看了看,虽然书上讲的很浅显,发现自己貌似瞎蒙还蒙对了思路,呵呵。正好凑巧的是,刚刚看了一点点,网上就有个人问怎么做Vertex Cover的问题,正好让我来做做练习。

\begin{enumerate}
\item 打印任意合法的n对括号:
\end{enumerate}
\lstset{language=java,label= ,caption= ,numbers=none}
\begin{lstlisting}
void printParenthes(int N, int left, int right, stack<char> &stk) {
    if (left == N && right == N) {
        printStack(stk);
        return;
    }
    if (left > right){
        stk.push(')');
        printParenthes(N, left,right+1, stk);
        stk.pop();
    }
    if (left < N){
        stk.push('(');
        printParenthes(N, left+1, right, stk);
        stk.pop();
    }   
}
\end{lstlisting}

\begin{enumerate}
\item Vertex Cover(NPC问题),图G中找一个顶点的最小子集,覆盖图的所有边。
\end{enumerate}
\lstset{language=java,label= ,caption= ,numbers=none}
\begin{lstlisting}
int current_k = N;  //global
void VC(int k, int start_v){
    if (all_edge_covered(G) && k < current_k) {
        current_k = k;
        return;
    }
    if (k == current_k - 1) return;       //剪枝
    for(; start_v <= N; start_v++) {
        if (!edge_list[start_v].empty()){ //剪枝
            list<int> temp_edge_list = edge_list[start_v];
            clear_edge(start_v,G);
            VC(k+1, start_v+1);
            if(curent_k == k+1) return;   //剪枝
            reset_edge(start_v,temp_edge_list,G);
        }
    }
}
\end{lstlisting}

想了想,其中的for循环其实是不必的,对于解空间树是子集树的问题,只需要考虑《当前顶点“选”“不选”》两个情况改进后的算法是:
\lstset{language=java,label= ,caption= ,numbers=none}
\begin{lstlisting}
void VC2(int k, int start_v){
    if (k<current_k && all_edge_covered(G)) {
        current_k = k;
        return;
    }
    if(k >= current_k - 1) return;  // 剪枝
    if(start_v == N) return; //没有下一个顶点了
    if(!edge_list[start_v].empty()){ //如果
        list<int> temp_edge_list = edge_list[start_v];
        clear_edge(start_v,G);
        VC2(k+1, start_v+1);
        if(curent_k == k+1) return; // 剪枝
        reset_edge(start_v,temp_edge_list,G);
    }
    VC2(k, start_v+1); //不选start_v这个顶点
}
\end{lstlisting}

\url{http://www.mitbbs.com/article/JobHunting/31502231_3.html}

\section{题:}
\label{sec-26-4}

\begin{enumerate}
\item 我们知道,从一个数组里找一段(连续的)子数组求最大和,是一道经典的面试题,方法很简单,只要O(n)的时间。把这个问题变一下,假设是一个循环数组呢?找一个size<=n的子数组with最大和。
\end{enumerate}

分析,很容易想到第一步,找个地方把循环数组切断,回到了原来的问题,然后
在考虑一下额外的情况。额外的情况就是:有可能最大和的子数组是跨越了切断
点的?这种情况的最大和怎么求呢?一个naive的方法能做到O(n),但是需要
O(n)的空间。巧妙的解法就是,注意到所有数的和是固定的,考虑切断后的非循
环数组,找一段从首开始+一段从尾开始的两个子数组with最大和,等价于找一
段子数组with min sum.

总结,要擅长利用等价性转换问题,从而将新的问题转变为一个已知有好solution的旧问题。利用已知的经典问题来解决新问题,可以说是面试题目中相当重要的一个技巧

\begin{enumerate}
\item largest rectangular problem:问题是这样的,一个N×M的棋盘,上面的数字要么是1,要么是0,那么要:
\end{enumerate}

a)最大的一个正方形全是1填充,

b)最大的全是1的矩形。

a)是用动态规划做,虽然方法也很好,但是这里就不提了。

b)问题感觉上要比a难很多,为什么呢,因为rectangular比square有更大的自由度。不好用DP来做,分冶也不合适。

这题的奥妙就在于,利用经典问题。什么经典问题呢?其实是另外一道面试题,其本身也是有一定难度的题,题目是:给你一个统计直方图,假设每根柱子都是单位宽度,从图的最左边一个紧挨一个排到图的最右边,求在这个图里找到一个最大矩形,它不跟任何直方柱相交(边缘接触是允许的)。为什么提起这个题呢,故事是这样的,我之前没有做出O(N*M)解法的largest rect题,后来有一天遇到了这个直方图的题目,找到了很漂亮的O(N)解法,猛然回顾起那道largest rect的题,这次就很轻松的搞定了。

3(鸣谢mittbbs jobhunting版上的一位面试官贡献自己出的题)有n个房间,小偷每天偷一间,偷的规律简单说就是随机行走,如果今天偷了第i间屋子,明天有一半的几率偷i-1,一半的几率偷i+1,注意如果刚好偷到了边界上,那么第二天只有唯一的选择。如果你是警察,你只能每天选择一个房间蹲守,并且贼的手段相当高明,偷了一个房间后,没有任何人能发觉该房间是否曾经被偷过。

提示:奇偶性。总结:注意观察题目中隐含的性质。

\begin{enumerate}
\item wild card匹配+搜索:假设你有一个dictionary(原题中是URL集合),你要搜到到所有与 *a*bc*d 这样的输入所匹配的words。这里,*是通配符,可以当成是任意个任意字符(包括空),怎么 预处理+搜索?如果输入是 ???a???b??cde 这类呢? ‘?’代表单个任意字符。如果输入是? *的混合呢?
\end{enumerate}

\url{http://www.mitbbs.com/article/JobHunting/31502229_3.html}

有m个nuts, n个bolts,规格大小都不相同

只能nut和bolt之间比较

怎么把他们排序?要求复杂度最小

\url{http://www.mitbbs.com/article_t/JobHunting/31502045.html}

题目其实都不难,behavior问了你最喜欢的CS的东西是什么,我就说算法,然后他还居然提了下我简历上量子计算,估计他不会怎么感兴趣,我就只简单提了一点。

然后coding题都是很基本的,

一个串in place删除某些字符,code完了后,反过来,一个串在某些地方插入字符,期间我脑子短路了一阵子,还好过了一些时间后接上来了。。。

然后是个超级老题,数组里面找唯一一个出现了奇数次的整数,我这次很诚实,
直接说我知道这类trick的。。。不过他还是让我接着说了一下笨办法怎么做。

接下来又是一个老题,楼层扔鸡蛋问题,这个我前面的日记里面都贴过的,所以除了表述上可能有些不清楚外,算法本身肯定是optimal的了。接下来问了面试官几个 cliche的问题就结束了,至少表面上面试官还是比较满意的。。。

整个面了下来居然没有问design的题目,也不知道是幸运还是不幸(一次都还没被问过,缺少实战经验)。。。

\url{http://www.mitbbs.com/article/JobHunting/31502227_3.html}

面的一般,关键感觉那面试官比较冷,问他也不说对错,让人挺郁闷。
\begin{enumerate}
\item 如何寻找二叉树(not binary search tree)的least common ancestor.
\item 如何测试一个计算器。
\item 如果你想打电话面试一个人,拿到那个人的简历,却发现他没有提供电话号码,你
\end{enumerate}
能想到几种方法找到他的电话号码。

最后一个问题还是挺好的,不知道大家能想到几种方法。 第一个问题我知道如何做,
就是找到从root到两个节点的path, 然后比较path就可以了。但他问的很细,可能我讲
的不是很清楚. 我现在想像这样具体到代码的问题,咋们能不能说在电脑上写好了发到
他邮箱,然后再解释啊? 不然像这个问题,一行一行解释真地挺难的。

\url{http://www.mitbbs.com/article/JobHunting/31502699_3.html}

very easy, but I think I have to say goodbye
\begin{enumerate}
\item deadlock's four condition
\item what's virtual memory
\item how to increase the virtual memory
\item when should we maintain v2p page table, when should we maintain p2v page table.
\item what's mmap \& lazyloading. when should we use mmap to allocate memory
\item multi-level page table
\item implement a stack with lock to ensure thread safe
\end{enumerate}
\url{http://www.mitbbs.com/article/JobHunting/31356292_3.html}

大部分的题都是板上贴过的,再贴贴吧:)
\begin{enumerate}
\item C++和C\#的最大区别?
\item 问到了C++和Java在Runtime Environment(没太听清,可能是这个词)的区别。后来我问问了,好像是在virtual function上的区别。我答得是Java里所有函数都是virtual函数,C++要显示标明。
\item TCP和UDP的区别前面3个应该是根据我简历来问的,其实我Java和C\#也没用太多:(
\item 给定一个int数组和一个int变量叫sum,返回是否有数组中的2个数的和等于sum。 这个问题我最开始没处理好正好数组里有一个数等于sum/2的情况:(
\item n级台阶,一次可以上1级或2级,有多少种上发 老题目了,fibonacci number
\item reverse the words in a sentence, but don't reverse the words.
\end{enumerate}

\url{http://www.mitbbs.com/article/JobHunting/31344095_3.html}

\begin{enumerate}
\item Given a random generator which can generate integer number from 1 to 5 with uniform probability. how to generate number from 1 to 7 with uniform probability.

\item Find the shortest path to convert one string to another using the minimum edits with each transformation string being a valid dictionary word in a dictionary.
\end{enumerate}

for example: for->fork->ford->word->sword

\url{http://www.mitbbs.com/article/JobHunting/31429703_3.html}

how do you dynamically allocate space for a two dimension array in consecutive bytes? 

should be easy

\url{http://www.mitbbs.com/article/JobHunting/31432089_3.html}

given N points in a place with their (x,y) co-ordinates. Find two points with least distance between them.

\url{http://www.mitbbs.com/article/JobHunting/31437667_3.html}

plz implement a non-recursive post order tree traversal. I think this is difficult. It is kinda simple for pre-order and in-order, but post-order is tough.

\url{http://www.mitbbs.com/article/JobHunting/31455707_3.html}

How do you measure context switch time in OS? any ideas?

\url{http://www.mitbbs.com/article/JobHunting/31465291_3.html}

ou have a data structure of integers, which can be negative, zero, or 
positive, and you need to support an API with two public methods, insert(int
) and getmedian(). Describe a data structure you would use to support this 
API and describe the running time of the two methods.

\url{http://www.mitbbs.com/article/JobHunting/31472621_3.html}

How to sort an array with only \{0, 1, 2\} possible values in O(n) without extra space?

Ex: an array \{0, 1, 2, 2, 1, 0\}

\url{http://www.mitbbs.com/article/JobHunting/31472623_3.html}

given a string, how to do a string rotation without using extra memory?

\url{http://www.mitbbs.com/article/JobHunting/31473311_3.html}

Q1) Given a list of characters and an int which is the distance
between the same characters. 

Eg: input- aaaaabbbbcc and distance as 2. One of the outputs can be- aabaababbcc

Come up with an algorithm and Code it.

Q2) Find out if two inputs are Anagrams with HUGE HUGE input (like thousand of terabyte) 

Q3) Given lots and lots of points in a 2D space find all the line with most points on it.

\url{http://www.mitbbs.com/article/JobHunting/31480153_3.html}

dictionary is given. You have a word which may be misspelled. How will you check if it is misspelled?

\url{http://www.mitbbs.com/article/JobHunting/31485125_3.html}

\chapter{bloomberg's phone interview question}
\label{sec-27}
\section{}
\label{sec-27-1}
\begin{enumerate}
\item How to implement garbage collector ( what data structure)
\item How to implement c++ smart pointer
\item Pro and Con of multi process and multi-thread
\item How many stack/heap does a multi-thread program with 10 threads have?
\end{enumerate}
10 stacks? 1 heap?
\section{}
\label{sec-27-2}
1.为什么加入bloomberg?
2.introduce your experience。
3.比较c++和java的区别。内存管理,garbage class\ldots{}.compiler,编译的作用?(
gcc)(不懂java。java貌似编译了以后各个平台上都可以运行吗?virtual machine?C
++编译
了以后的汇编语言只能在特定的系统上运行?)
4.C++中什么function call delete a object? destructor
5.谈到了shallow copy deep copy.
\begin{enumerate}
\item 
\end{enumerate}
\lstset{language=java,label= ,caption= ,numbers=none}
\begin{lstlisting}
int main()
{
  return main();
}
\end{lstlisting}
will this compile? will this run? 
\begin{enumerate}
\item 利用一个写好的函数putchar(char A)which prints out the character you
\end{enumerate}
entered. 写一个putlong (long A).只能调用putchar,不能调用其他任何函数(可以用
STL,但是不能用STL中的函数)。后来还问到了如何test你的程序。这一道题折磨了我
15分钟。
only-putchar/
所有问题不告诉你回答正确与否。他没有固定的几道题要问,你谈到了什么就往深里面
问。今天有50多人面试,各个背景的都有。CS/EE的我感觉就问编程,非CS/EE的问智力
题。再有就
是通知巨突然,周四晚上收到邮件,让周六早上面试。。。
体会:是不是编程牛人,一写程序就知道。。。。这次又去当炮灰了。

\url{http://wuhrr.wordpress.com/2007/11/09/how-to-print-a-long-integ}

\url{http://www.mitbbs.com/article_t/JobHunting/31500097.html}

\chapter{hedge fund}
\label{sec-28}
职位是 junior financial engineer, 公司是一hedge fund,其实面完就感觉不太好,
一共见了6个人,有两个人问得技术问题答得不太好,也怪自己事先准备面试下的功夫
不太到家,准备得重点没有把握好。以下是一些能想起来的问题:
1.C++ 中的virtual destructor是啥? 为啥要用? 
2.quick sort, merge sort的复杂度. 
3.Structure 和class的区别是什么? (我晕,这个我居然给答反了)
4.关于C++ 处理异常的方法 . (基本上一头雾水)
5.Monte Carlo method in american style option pricing. (我说的用least 
square regression method,blah\ldots{}\ldots{})
\begin{enumerate}
\item Int$_{\text{0}}^{\text{T}}$ W(t) dW(t) ( 一看见这个,贼激动阿,熟悉的ito' s formula)
\item Stonivich intergral 是啥? 为什么用Ito's 不用 stonivich? (不知道拼得对不
\end{enumerate}
对)
\begin{enumerate}
\item 一个国家所有的人如果生了一个男孩以后就停止生育,生了女孩以后就继续生,直
\end{enumerate}
到生出男孩才停止生育,问多年以后男孩多还是女孩多? (要联系上stopping time的
概念)。
\begin{enumerate}
\item 什么是AR model? 啥时候用AR model?
\item American option 的up bound? (我说是stock price,被直接鄙视了,说更精确的
\end{enumerate}
,只好答没有研究过,当时一头雾水)。

还有就是,关于自己的简历上面的Project 工作经历,一定要熟练再熟练,有些人问得
那叫一个细啊,而且基本上我所有的Project都被人问到了。这次面试的前4个人主要问
计算机和金融方面的技术问题,第5个HR,问些personality的问题,最后是hiring 
manager,因为之前电话面试过我,就没有问问题,简单聊聊。整个面试花了5个小时,
雷死了,脑子到后面都已经不转了。虽然结果让人遗憾,不过就当是学习了,贴点信息
和大家共享下,希望自己能早日找到工作,也希望还在努力找工作的XDJM们再加把劲,
大家一起加油。 
\url{http://www.mitbbs.com/article/JobHunting/31406731_3.html}

CS方向,希望对大家准备面试有帮助
\begin{enumerate}
\item 用stack class来实现queue,具体用几个stack不限。完了以后问怎么实现thread safety,然后是怎么测试。
\item 实现strstr(str1, str2),如果str2是str1的子串,返回true,否则返回false。实现完了以后问如何测试。
\item 给定一个integer array with both positive and negative numbers,return a contiguous subarray with the largest sum. 我本来想用dynamic programming实现,但面试官希望按照他的一个更heuristic的思路来解,最后勉强搞定。
\item 给定一个排好序的linked list,删除其中所有的重复元素。比如给定1->2->3->3->4->4->5,返回1->2->5。给定1->1->1->2->3,返回2->3。看起来简单,一边写一边发现许多细节需要小心应对,好在最后搞定。
\item 给你三个烤箱,每个烤箱可以同时烤两片面包,需要的时间分别是3分钟,4
分钟和3分钟。但第三个烤箱有一个slot出了点问题,每次只能烤面包的一面。
所以这个烤箱三分钟后只能算烤好一片半面包,你需要把那半片翻个面,在
同一个slot里再烤一次才算一片完整的。现在给你这三个烤箱,问烤好21片
面包最少需要多少时间?如果是2100片呢?如果是任意给定的N片,要求O(1)
时间内给出最少需要的时间。
\item 给你三根棍子,每根都需要一个小时才烧完,但每根燃烧的速度都不一样,也不均匀。问只有这三根棍子和火柴,如何精确的得到1小时45分钟的计时?
\item 在一个party上,每个人可能认识别人,也可能不认识。现在其中有一个人是名人,定义就是所有的人都认识他,但他不认识其余的任何人。现在要求你去找出这个名人来。但你只可以通过一个方法,就是问A是不是认识B,回答是表示A认识B,不是表示A不认识B。你可以任意去问这样的问题,问最少需要多少次能找出这个名人?思路有了之后要求写代码实现,可以调用knows(A, B),代表上面的那个问题。实现完了以后问如何测试
\item 测试copy这个命令。然后自己问了一些clarifying questions,搞清了实际
上是copy src dest。src可以是文件,也可以是目录。dest可以存在,也可
以不存在。
\end{enumerate}
\url{http://www.mitbbs.com/article/JobHunting/31410833_3.html}

OO设计题,
\begin{enumerate}
\item 怎么做一个十字路口的traffic light.
\item 怎么不用recursion 做二叉树in order 遍历。
\end{enumerate}
\url{http://www.mitbbs.com/article/JobHunting/31421129_3.html}

\begin{enumerate}
\item Write a function that returns a node in a tree given two parameters: pointer to the root node and the in order traversal number of the node we want to return. The only information stored in the tree is the number of children for each node.
\item Input a message and a text, find if the message can be composed by
the text. If the text is in a magazine (two pages/a paper), how to design an algorithm?
\end{enumerate}
\url{http://www.mitbbs.com/article/JobHunting/31422009_3.html}

\begin{enumerate}
\item When casting an object of a polymorphic class from a base calss type, which one of the following castsperforms the task only if the cast is valid?
\end{enumerate}
a. static$_{\text{cast}}$
b. (void*) 
c. dynamic$_{\text{cast}}$
d. const$_{\text{cast}}$
e. reinterpret$_{\text{cast}}$

\begin{enumerate}
\item 
\end{enumerate}
\lstset{language=java,label= ,caption= ,numbers=none}
\begin{lstlisting}
class A {
public:
    void f();
protected:
    A() {}
    A(const A&) {
    }
};
\end{lstlisting}
why are the default and copy constructors declared as protected?

a. to ensure that instance of A can not be created via new by a more derived class

b. to ensure that instance of A can only be created by subclasses of A

c. to ensure that isntance of A can not be copied

d. to ensure that A cannot be used as a base class.

e. to ensure that A cannot be instantiated on the stack

\begin{enumerate}
\item 
\end{enumerate}
\lstset{language=java,label= ,caption= ,numbers=none}
\begin{lstlisting}
template<class T1; class T2; class T3>
int Product(T1 a, T2 b, T3 c) {
    return a*b*c;
}
\end{lstlisting}
what is wrong with the sample code above?

a. templates must be class definitions

b. the template parameters should be separated by commas.

c. the template definition is missing a pair of braces.

d. template parameters must be pointer types.

e. the * operator has not been defined for T1, T2, and T3.

\begin{enumerate}
\item 
\end{enumerate}
\lstset{language=java,label= ,caption= ,numbers=none}
\begin{lstlisting}
class FOO {
    char * buf;
public:
    Foo (const char *b = "default") {
        if (b) {
            buf = new char[std::strlen(b) + 1];
            std::strcpy(buf, b);
        } else         
            buf=0;
    }
    ~Foo() {
        delete[] buf;
    }
};

Foo func (Foo f) {
    return f;
}
\end{lstlisting}
when the function fun is called, the program may crash or exhibit
unexpected behavior, what is the reason ofr this problem?
a. the destructor may attempt to delete the string literal "default"

b. the destructor needs to check that the value of buf is not 0.

c. the class does not allocate a long enough buffer.

d. the function needs to return Foo\& instead of Foo.

e  the class needs to specify a copy constructor and assignment operator.

\url{http://www.mitbbs.com/article/JobHunting/31426509_3.html}

1.请书写一个程序,将整型变量 x 中数字左右翻转后存到另外一个整型变量 y
中,例如 x = 12345 时,y为 54321,x = ‐123 时,y为‐321。其中 x 的个位
不为 0。   
\lstset{language=java,label= ,caption= ,numbers=none}
\begin{lstlisting}
void reverse (int x, int* y);
\end{lstlisting}
~ (1)  请实现该函数,以上函数原型是用 C语言写的,你可以用你熟悉的语言; 

~ (2)  请写出一段代码验证该函数在各种情况下的正确性。 

2.对集合\{1, 2, 3, …, n\}中的数进行全排列,可以得到 n!个不同的排列方式。现在我们用字母序把它们列出来,并一一标上序号,如当 n=3 时:  0.123   1.132   2.213   3.231   4.312   5.321 现在,请书写一个函数 void print (int n, int k), (函数原型是用 C语言写的,你可以用你熟悉的语言)在已知 n和序号 k 的情况下,输出对应的排列,并简要阐述思路。 

3.一维数轴上有 n 条线段,它们的端点都是已知的。请设计一个算法,计算出这些线段的并集在数轴上所覆盖的长度,并分析时间复杂度。例如,线段 A 的坐标为[4, 8],线段 B 的坐标为[1, 5.1], 那么它们共同覆盖的长度为 7。 请尽量找出最优化的算法, 解释算法即可,不必写代码。

\url{http://www.mitbbs.com/article/JobHunting/31428195_3.html}

Given a sorted integer array and a number, find all the pairs that sum
up to the number.

这个很简单,但现在多了一个条件What if the array is sorted by absolute value, for example \{1, -2, 4, -9\}, find the answer in O(N).这样有什么好的思路么?

\url{http://www.mitbbs.com/article/JobHunting/31430593_3.html}

How do you find sequences of consecutive integers in a list that add to a particular number.Array里面正负数都有.这个能在O(n)时间内解决吗?

\url{http://www.mitbbs.com/article/JobHunting/31431861_3.html}

A m*n matrix of integer, all rows and columns are sorted in ascending
order. Find the most efficient way to print out all numbers in
ascending order. 

\url{http://www.mitbbs.com/article/JobHunting/31434325_3.html}

一次面世Google,问到hash table是怎么实现的。我说了一个取尾数(round)的方法,他说这个方法很navie,工业界一般用其他的方法,比方说STL的map。我想了半天没有想出来,到这里问问。hash table具体怎么实现的啊?

\url{http://www.mitbbs.com/article/JobHunting/31434401_3.html}

49 辆赛车. Assume for each one, it travels the track in the same amount of time every time. Also assume no two finish the track in the same amount of time. Suppose you have 7 tracks, but no timer. Design races to find the 25-th fastest with minimal number of races.

\url{http://www.mitbbs.com/article/JobHunting/31434523_3.html}

How do you know the bloomberg? 

What position do you expect? 

What language do you want to answer with? (I choose C.) 

What kind of questions do you meet for the online assessment?

what is static in C? how is it implemented by the compiler?

write the definition of a function that returns both the max and min.

why do you use the condition variable?

how to implement a lock?

Under what condition must you use linked list instead of array?

what data structure can you use to store elements dynamically and
access them efficiently? The complexity of finding any element in a
linked list in the worst case. multi-thread library programming: did you write your multi-thread library 
with p-thread? is there any problem you have with you library?

did you do your projects on linux? If you want to find a string in a file, 

what command should you use?

do you know vector in C++?

a question about real-time programming (I forgot)

what is buffer overflow?

一些问题是针对我的简历里面提到的内容,所以,简历里面的内容要尽量的吃透。

\url{http://www.mitbbs.com/article/JobHunting/31434685_3.html}

Given two classes:
\lstset{language=java,label= ,caption= ,numbers=none}
\begin{lstlisting}
class B {
public:
    B(args_1);
    B(args_2);
    // and many constructors with different arg lists
};

class D : public B {
public:
    D(args_1) : B(args_1) {}
    D(args_2) : B(args_2) {}
    // and many constructors with different signatures similarly implemented
    // some additional stuff specific to D
};
\end{lstlisting}

Assume that the arg list for B's constructors are quite long and may be
revised pretty often in the future, in which case D's constructors have
to be recoded correspondingly. Duplicating the update by copy-and-paste
will certainly work here. Can you propose a better way so that the
update can be done in one place without copy-and-paste duplication?

\url{http://www.mitbbs.com/article/JobHunting/31434891_3.html}

Given a large string (haystack), find a substring (needle) on it.感觉这道题不就是scan一遍吗?有什么time and space complexity上面的trick吗?

\url{http://www.mitbbs.com/article/JobHunting/31435419_3.html}

准备了很久,看了很久算法的书。。 结果被问了一个怎么 optimize memcpy()..傻眼了。。碰到了女老印,倒霉~~~~

\url{http://www.mitbbs.com/article/JobHunting/31435587_3.html}

给一个substr,如何判断它在不在给定的str里面。substr有两个新的符号可能在里面:

(1)* : 0-n个任意字符

(2)? : 1个任意字符

太紧张了,所以面试者简化了题目,说去掉“?”,然后让code和测试:基本框架出来了,但是好多特殊情况没有处理到,比如substr以“?”起头。后来又问如果加入“*”有没有思路,刚说了两句就out of time了。

\url{http://www.mitbbs.com/article/JobHunting/31436721_3.html}

给定 X[1..n] and Y[1..m] 两个 arrays,已经sort好了. 如何找到X <Union> Y的median?我说用merge sort,要O(m+n/2)时间。面试官明显不满意。这个已经 lineal了?难道还有更快的?

\url{http://www.mitbbs.com/article/JobHunting/31437417_3.html}

given a 32 bit number N and 2 numbers(A \& B) that determine 2
different bit positions of N how do you make all the bits between A
and B equal to another giveninteger k. given (A,B is in the range [0
to 31] and k<=2$^{\text{(B-A+1)}}$ ( so that k fits between B-A+1 bits). Give an
O(1) solution forth is e.g if N=9 ( 1001) ,A=0 ,B=2,K=5(101 then the result should be 1101 (1.e 13)这个题是什么意思啊?

\url{http://www.mitbbs.com/article/JobHunting/31437907_3.html}

在做careercup上面的题目, 有两个问题没有看懂, 希望有人指点下

1 一个BST, 给定一个值, 打印出所有的path,使path上所有节点的值等于给定值;

2 一个tree, 如何高效的找出最长的path? 

这都是amazon的题目吧

1.sum of all nodes in a path  = givenValue

2.\url{http://www.careercup.com/question?id=87897}

\url{http://www.mitbbs.com/article/JobHunting/31441709_3.html}

第一道是写一个函数,两个参数(String prefix, String s), 返回true如果s
有prefix

第二道是写一个函数,两个参数(int[] a, int sum), 找出数组里加起来是sum的几个数我第一题算是答出来了,第二题没做完,没有好的思路。。。

\url{http://www.mitbbs.com/article/JobHunting/31446979_3.html}

\chapter{Adobe}
\label{sec-29}
Went to Adobe to interview a Senior SW Engineer position, 总的interview的不错, 但被下面问题问倒了,让回去想想,  

Q1: "We need to compare thousands text files with each other, they are not big, 
less than 100K each. They are in a directories tree, with a few levels of 
subdirectories,  how to speed up the comparing process ?"
My answers: We can read them all of these files into memory once so that we 
can reduce the number of diso I/O.
[Feedback: That is a good appoach].

Q2: How to read these files into memory (on MS Windows platform ) ? how do 
you maintain directory structure in memory ?
My answer: I talked some garbage \ldots{}.

Q3: If someone already wrote the code in slow way, read each file from disk,
do some thing, close the file, read another one, etc.  Can you make a "
portable layer API" libary so that with minimal effort, old code can still 
work but much faster ? (of course, we need to recompile the code).

Please help with Q2 and Q3, thanks a lot.

\url{http://www.mitbbs.com/article/JobHunting/31448285_3.html}

今天把M的onsite给拒了,实在没有时间面这么多company,又不想浪费别人的时间。不过心里还是觉得有点可惜,啧啧。贴一下M的经历吧。On campus就一轮,30分钟。Interviewer是个老中,一上来看我resume,问为啥phd了还来面sde。然后开始问resume上的东西,我借机会sell了一下自己。

Technical问题只有两个:

1)Coding题非常old了。两个string找最长common substring。这个当场肯定code不了subffix tree。于是就用暴力的方法,三下五除二搞定。然后问complexity,如何改进,bla bla bla。竟然忘记了说可以用DP,低级失误啊。不过面试官还算满意。

2)你认为bing有什么可以改进的(我投的是bing)?你research做的东西有没有可以apply的?

\url{http://www.mitbbs.com/article/JobHunting/31451397_3.html}

今天又做coding面试了,上来就要写个函数 返回二叉排序树的第k个最小的node。我写了一半,感觉不对劲。请大侠赐教。

\url{http://www.mitbbs.com/article/JobHunting/31451705_3.html}

you are given a M x N matrix with 0's and 1's find the matrix with largest number of 1,

\begin{enumerate}
\item find the largest square matrix with 1's

\item Find the largest rectangular matrix with 1's
\end{enumerate}

\url{http://www.mitbbs.com/article/JobHunting/31452521_3.html}

有几个同学问面筋,不太记得起来,很多版上是有的,所以觉得那些面你的人水平挺一般的,下面贴两个印象深刻的:

1。这道题被好几个不同的公司面到过:Fibonacci数列,一般让你给一个recursive的版本,然后写个iterative的版本,然后问有没有更快的可能性。我记得以前在某个版讨论过,参考wiki:这样的方法,可以在O(log(N))的时间和O(1)的空间复杂度内算好。要写程序的话,用
类似下面的方法:
\lstset{language=java,label= ,caption= ,numbers=none}
\begin{lstlisting}
Matrix2x2 F[][2] = {{1, 1}, {1, 0}}, Fn[2][2] = {{1, 0}, {0, 1}};
while (N) { if (N & 1)     mul(F, Fn, Fn);    // Fn = Fn x F;
 mul(F, F, F);          // F = F^2; N = N >> 1;
}
\end{lstlisting}

2。另外一题很简单,但是蛮tricky的。How to test if a number "a" is
power of 2 return (a-1) \& (a) == 0;网上经常有问怎么样判断一个数里面有
多少个1的位数,这个只是其中一个最简单的特

\url{http://www.mitbbs.com/article/JobHunting/31452533_3.html}

于完成了F公司历时2个月的所有interview,总算可以松口气了,据称他们下周一开会讨论,希望最终会修成正果。来说点经历吧。多亏好朋友Z帮忙forward resume,很快就来了第一轮phone interview。编程题还有点老:

[Coding Q1]: Given an array A, output another array B such that B[k]=productof all elements in A but A[k]. You are not allowed to use division.其实这题interview之前在本版JHQ看过,可是当时看的题目太多,没有去想solution。所以刚开始听到这题还surprise了一下。我觉得这个不能用除法的限制太无聊了(建议改个problem来问这个algorithm),于是忍不住问why not division,顺便拖延一下时间想算法。面试官说除法慢\ldots{}显然不是什么很convincing的理由,我说那乘法也慢啊。说完我已经想到怎样做了,于是顺利过关。接着就来了比较衰的第二轮,题目是这样的:

[Coding Q2]: You are given a string e.g."face" and a set of mutation rules, e.g. a->@, e->3, e-E. Print all the possible strings that can be generated by the rules, e.g. f@c3, fac3, etc.其实就是BFS再加上hash table判断是否重复print。马上就想到algorithm,面试官说好,你开始写吧。然后问题就来了,太久没写c++忘了hash table的函数定义。好像依稀记得hash table还有几个版本,想了一会没想起来,又不好意思问,汗!最后还是忍不住问了,他说你随便给个函数名和接口吧。最后磕磕碰碰总算把程序写完了,却给人留下了很不好的印象,感觉写程序很不熟!据说最后这个人给了我一个borderline,还算好,没把我fail掉。真惭愧啊,可怜我还是写c++起家的\ldots{}因为第二轮不太理想,本来应该两轮过后就onsite,结果hr来信说要第三轮phone,还很好人的说We do have three phone interviews at times. We are constantly evaluating our process so I apologize for the change. 第三个面试官又临时换人,最后居然是同系师弟,不过之前没见过面,不然可以套近乎了。。。言归正传,换人大概还是因为他们组想看看我match与否。他问的都是machine learning,风格和原来完全不同,还多多少少有点surprising的。

[Coding Q3]: Implement one step of decision tree which splits the node into two subtrees.之后还讨论了一些learning的问题,我问了他们用的technique,有什么存在问题等等,相谈甚欢。

关于onsite,因为签了NDA,不方便透露题目。请大家也不用发信来问了,做人还是要讲信用的。只看面经的各位看官可以略过以下了。。。至于那个onsite可谓一波三折。本来订了机票周四晚上到sfo,周五中午onsite,挺好的schedule。结果某airline居然机件故障,把飞机拖去修了几个小时,又不肯调其他飞机来,白白miss掉了从vegas飞sfo最后一班航班,被迫在vegas住了一晚。下了飞机都晚上12点了,随便找个airport旁边的hotel住下,改了第二天最早一班机。结果第二天又晚点两个小时!据称SF大雾,traffic control\ldots{}折腾了半天到了F都周五下午两点多了,又累又紧张。连HR也只好说:it's hard to visit us\ldots{}不过不管那么多,灌了杯coffee就上阵了,结果还好,没有想象中intensive,也发挥自己的水平了. 因为onsite去的太迟,没见到manager,HR又说schedule TWO more follow-up,其中一个是manager。OMG,我说好,那就back-to-back吧。上周终于面完真是relieved啊,前后5轮,历时2个月。

关于面试的经验教训,我的感觉是

1)F的interview是比较严谨的,phone interview就要candidate在white board
上把code写出来,不是说说算法就算了,detail也问得很仔细。因为他们要求员
工follow整个project,从idea,到algorithm,到implementation,而不是自己
想个东西出来让别人写code实现就完了。各位像我一样平时写research matlab
code多于写c++的phd要注意多练练手了,小心阴沟翻船。

2)Never give up无论interview多么不顺利。不要被外界的不利因素distract自己,该准备什么就好好准备,我相信life has miracles.

PS: 貌似F的同学们也会上来job hunting版。文中若有冒犯之处,请多多包涵。F的S同学,你那题大概说了也不要紧吧,没理解背下来也没啥用的。F的Y同学,我没有泄漏你的面试题,以后还可以继续用,哈哈。

\url{http://www.mitbbs.com/article/JobHunting/31452725_3.html}

5。Given a graph (any type - Directed acyclic graph or undirected
graphs with loops), find a minimal set of vertices which affect all
the edges of the graph. An edge is affected if the edge is either
originating or terminating from that vertex. The time should be less Q(n$^{\text{2}}$)这个题就是最小顶点覆盖问题吧?或者是我对最小顶点覆盖问题理解有误?或者对这题理解有误?

\url{http://www.mitbbs.com/article/JobHunting/31452961_3.html}

00*100部分有序矩阵数组的排序, 有100个有序数组(从小到大),每个里面有100个数。设计一个算法合并这个一百个有序数组,中间步骤只允许多申请一个大小为100个数的空间(也就是一个数组的大小)。

\url{http://www.mitbbs.com/article/JobHunting/31453089_3.html}

\begin{enumerate}
\item How to call C++ code in C? How to call C code in C++?

\item In which three cases, initialization list has to or is preferred to be used for a constructor?
\item Can we design singleton by setting all the data member and method of a class to be static?
\item Is overloading allowed in C? If not, how to differentiate them?
\item Default methods that are generated by a class in C++.
\item Difference of struct and class in C++
\item Given a class has first name, last name, SSN and etc.Need to query according to first name, first name + last name, what STL should be used? (If map/multimap, what should be the key?) How the query should be? How to query all the first name that initialed as "J"?
\item Meaning of static in C and C++
\item Meaning of inline in C++; where should it be used?
\end{enumerate}

几天上午一个面试的问题。有些东西没用过,虽然以前看过,但是还是没有答出来;都去准备其它的去了,没想到全是问c/c++的问题。还是有些不服气,move on。

\url{http://www.mitbbs.com/article/JobHunting/31454759_3.html}

\chapter{cloud service}
\label{sec-30}
面的是ELASTIC COMPUTING CLOUD组的SE,我是做网络的,对他们的这个cloud service有些兴趣,以为会问算法和系统的问题,结果问了一个OOD的问题,说一个大楼,10层,4个电梯,怎么设计类来实现这样一个系统? 题目career cup上有,不过没想到他会问这个,ECC又不是做应用的.刚好是我的弱项,一直做research,对算法和语言还算了解,对应用系统和设计那是一片空白.面的是一塌糊涂. 有要面amazon的参考一下.
Windows Live Experience 组:1.美国人。上来随便聊聊,然后出了个coding 题目 一个数组,找出第一个重复的数  我给了三种方法,最后用hash写的,然后问test case之类的

2.印度人  上来问我会什么C\#还是C++,我说C\#会的多一些。然后他上来问了四五个简单的 语法问题。正好我还都会,心理还窃喜着呢。coding 也很简单,给一个01字符串,转化成整数。写完后 test case。 第二个题目是两个函数互相调用,无限循环了,然我找出毛病,问怎么解决。  然后午饭跟这个印度人吃,随便聊聊。 就过了

3.欧洲人,不知道哪国。  女的, 人很好,跟她聊的最开心。coding 题目是个没见过的,double bytes string实现delete键功能。这个比较难解释,她开始也跟我解释了很长时间。 就是删除字符的时候如何确定是删一个字节还是删两个字节的问题。我给出算法,然后她有提示有哪些特殊情况要考虑,也做出来了。然后她就问我给一个一般的application 如何测试,又随便说了一通,结束了

4.美国人 senior test leadcoding 很简单,给一个句子,把里边所有的单词自身reverse然后给我看他们的产品,问我怎么测试。聊的也挺好

5.欧洲人 director面到这个人的时候,我都快累趴下了,都不想面了,实在是累。心理还想着,offer拿不拿得着无所谓,别把老子给累死了。(看来真得努力锻炼身体,不然面试都挺不住)题目也很简单,找1--100的素数。我就给了最简单的方法,然后我说要 check一些边界情况,他说不用了。然后让我做到他的椅子上,打开excel,问我怎么测设置字体这个feature。说完了问我有什么问题没有

给你一本dictionary,任意给你七个letters,让你找出包含这七个字母的、最长的单词。条件:可以pre-processing,这样每次给你不同的letters时,可以very effcient我当时想了好久也没给出完整答案。。。naive 的解法当然就是每次scan dictionary,每次 O(n)。。。pre-peocessing那就是建index,但index怎么建?怎么操作?

\chapter{FULL TIME SDE。}
\label{sec-31}
1.REVERSE LINKLIST.
2.给了N个数,值域[1,N-1],如何找出第一个重复的数
3.算POLYNOMIAL,比如5x$^{\text{4}}$+6x$^{\text{3}}$-7x$^{\text{2}}$-8=?
4.给一个URL,如何把空格这种字符转换成\%20这种
5.给一个LINKLIST,VALUE的指针指向其他NODE,复制他今天RECRUITER发邮件通知给OFFER了,漫长的两周。。。希望收回OFFER这种事不要发生。。
\url{http://www.mitbbs.com/article/JobHunting/31387663_3.html}

\chapter{ftware Development Engineer}
\label{sec-32}
问候之后,首先问了一下我的research, 让我具体的阐述我提到过的算法然后OOD的一道题,其实不难,但我感觉自己答得不好have a furniture class, some child classes like table, chair, etc.
they want to extend the class hierarchy, as there are wood table, steel table, wood chair, steel chair, and so on.
我首先给出class + interface的design:
\lstset{language=java,label= ,caption= ,numbers=none}
\begin{lstlisting}
furniture (table, chair, ...)
table ( woodtable extends table implements wood, steeltable extends table 
implements steel)
chair ( similar as table)
\end{lstlisting}
然而interviewer立刻指出这样的话,如果要加fire, 或者和重量有关的functions时,会有code duplication (因为wood可燃,steel不可燃, assuming fire function is defined in wood and steel interfaces). 我最后说那把wood和steel也改成class (c++ multi-inheritance), 这样可以avoid code duplication. Interviewer又问那么如何实现woodsteeltable? 我说就inherit woodtable and steeltable. 自己对这个问题总体感觉不好。大家有什么好的design?

后面的问题比较简单:

given a deck of cards, how to shuffle it?

having a web application, front-end, middle layer and database. How to
scale database to accommodate increasing traffic?

\url{http://www.mitbbs.com/article/JobHunting/31471911_3.html}

onsider a function which, for a given whole number n, returns the
number of ones required when writing out all numbers between 0 and n. For example, f(13)=6. Notice that f(1)=1. What is the next largest n such that f(n)=n?

\url{http://www.mitbbs.com/article/JobHunting/31471823_3.html}

请解释garbage collection?在garbage collection中,对circular reference的你怎么办?

\url{http://www.mitbbs.com/article/JobHunting/31470831_3.html}

栽在一道编程题上:Find a longest increasing subsequence in an integer array。问问题的人要求朋友拿出O(nlog(n))的算法,但朋友只给出了O(n$^{\text{2}}$)的dynamic programming的方法。其实我觉得给出dynamic programming算法足够进入下一轮了。那个O(nlog(n))的算法好歹也值当年一篇paper,而且貌似不是那么直观。电面就想出来不容易。不过多半是我段位不够,还不够Google的要求。或者朋友的dynamic programming其实错了(这道题要倒过来找,稍微绕一点点)。

\url{http://www.mitbbs.com/article/JobHunting/31473303_3.html}

Given n points in the form (x1, y1, z1)…..(xn, yn, zn), find the k closest points to the origin.

Given the same points as above, find the K closest points to each other.

\url{http://www.mitbbs.com/article/JobHunting/31473965_3.html}

\begin{enumerate}
\item one array filled with numbers from 1 to N, but one number is missing. wha
\end{enumerate}
t's the most efficient way to find the missing item? what about two or more 
numbers are missed?

\begin{enumerate}
\item find the repetative chars in a string and delete them

\item find the binary tree from its preorder and inorder traversal
\end{enumerate}

\url{http://www.mitbbs.com/article/JobHunting/31474331_3.html}

and(5) generates a random integer number between [1, 5], how do you 
generate a random integer number between [1, 7] when you can only call 
rand(5)?

\url{http://www.mitbbs.com/article/JobHunting/31476251_3.html}

Given an integer, print the closest number to it that is a palindrome

input: 1224 

return: 1221.

\url{http://www.mitbbs.com/article/JobHunting/31477969_3.html}

Given a value and a binary search tree. Print all the paths(if there exists more than one) which sum up to that value. It can be any path in the tree. It doesn't have to be from the root.我理解是这个path可以是其中任意一截,不用包括头尾

\url{http://www.mitbbs.com/article/JobHunting/31478003_3.html}

given a character string, print the number of occurence of each
charcater inorder. ie. if the string is "ceabcw", then you should
print something like:a 1 b 1 c2 e 1 w 1.she asked the possible data
strucutre to approach. I gave array, hashtable, and BST. she asked me
to use BST, and using no recursive. Also how to handle unicode.

然后问了一些测试题,让我测试她们的一个产品。细节忘了,总之她对我不满意。我也觉得基本没戏了。

第二个是个印度人。编程题:given a matrix(assume it is a bitmap), print all cells that is on.做的不好。后来问了一些测试题。

第三个是个白人。开始问测试的问题,回答得一般。因为觉得已经没有戏了,所以也不大有精神。编程题很简单,是实现阶乘。不过有个问题没有考虑到,就是overflow怎么处理。总之非常惨,第一次面试这么惨。也请给位说说自己的想法怎样解答那些问题

\url{http://www.mitbbs.com/article/JobHunting/31481069_3.html}

两个玩家, 一堆石头,假设多于100块, 两人依次拿, 最后拿光者赢, 规则是

\begin{enumerate}
\item 第一个人不能一次拿光所有的

\item 第一次拿了之后, 每人每次最多只能拿对方前一次拿的数目的两倍
\end{enumerate}

求先拿者必胜策略, 如果有的话

\url{http://www.mitbbs.com/article/JobHunting/31482015_3.html}

题目:
从一个string 变到另一个,比如"study"->"world" (字数相等),要求

\begin{enumerate}
\item 每次变一个字母

\item 每次改变后的string必须是一个词典里面能查到的英语单词,比如你不能把study变成atudy
\end{enumerate}

\url{http://www.mitbbs.com/article/JobHunting/31482527_3.html}

\chapter{Website Links}
\label{sec-33}
\url{http://www.mitbbs.com/article_t/JobHunting/31342084.html}

\url{http://www.mitbbs.com/article_t/JobHunting/31347263.html}

\url{http://www.mitbbs.com/article_t/JobHunting/31347264.html}

\url{http://www.mitbbs.com/article_t/JobHunting/31348374.html}

\url{http://www.mitbbs.com/article_t/JobHunting/31348607.html}

\url{http://www.mitbbs.com/article_t/JobHunting/31350186.html}

\url{http://www.mitbbs.com/article/JobHunting/31354737_3.html}

\url{http://www.mitbbs.com/article/JobHunting/31356298_3.html}

\url{http://www.mitbbs.com/article/JobHunting/31368921_3.html}

\url{http://www.mitbbs.com/article/JobHunting/31373641_3.html}

\url{http://www.mitbbs.com/article_t/JobHunting/31376671.html}

\url{http://www.mitbbs.com/article/JobHunting/31383513_3.html}

\url{http://www.mitbbs.com/article/JobHunting/31387661_3.html}
% Emacs 24.3.1 (Org mode 8.2.7c)
\end{document}