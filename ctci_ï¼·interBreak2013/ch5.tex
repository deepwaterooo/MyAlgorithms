\chapter{Bit Manipulation}
\small{}

\begin{description}
\item[5.1] You are given two 32-bit numbers, N and M, and two bit positions, i and j. Write a method to set all bits between i and j in N equal to M (e.g., M becomes a substring of N located at i and starting at j).

EXAMPLE:

Input: N = 10000000000, M = 10101, i = 2, j = 6

Output: N = 10001010100

给定两个32位的数,N和M,还有两个指示位的数,i和j。 写程序使得N中第i位到第j位的值与M中的相同(即:M变成N的子串且位于N的第i位和第j位之间)

输入: N = 10000000000, M = 10101, i = 2, j = 6

输出: N = 10001010100

Solution:

方案1:先将N中第0位到第i位保存下来(左闭右开:[0, i)),记作ret, 然后将N中第0位到第j位全清0([0, j]),通过向右移动j+1位然后再向左移动j+1位得到。 最后用上面清0后的值或上(m<<i)再或上ret即可。

代码如下:

\begin{lstlisting}[language=C++]
int update_bits(int n, int m, int i, int j){
    int ret = (1 << i) -1;
    ret &= n;
    return ((n>>(j+1)) << (j+1)) | (m<<i) | ret;
}
\end{lstlisting}

方案2:用一个左边全为1,中间一段全为0(这段的长度与m长度一样), 右边全为1的掩码mask去和n按位与,得到的值是将n中间一段清0的结果。 然后再与m左移i位后按位或,得到最终结果。

代码如下:

\begin{lstlisting}[language=C++]
int update_bits1(int n, int m, int i, int j){
    int max = ~0;  
    int left = max - ((1 << j+1) - 1);
    int right = ((1 << i) -1);
    int mask = left | right;
    return (n & mask) | (m << i);
}
\end{lstlisting}

C++中关于位操作,记录几点需要注意的地方:

一个有符号数,如果它的最高位为1,它右移若干位后到达位置i, 那么最高位到第i位之间全是1,例如:
\begin{lstlisting}[language=C++]
int a = 1;
a <<= 31;   
a >>= 31;   
cout << a << endl;
\end{lstlisting}

一个无符号数,如果它的最高位为1,它右移若干位后到达位置i, 那么最高位到第i位之间全是0,例如:
\begin{lstlisting}[language=C++]
unsigned int a = 1;
a <<= 31;    //a:1`后面带31个0`
a >>= 31;    //`a:31个0后面带一个1,即1`
cout<<a<<endl;    //`输出1`
\end{lstlisting}

无论是有符号数还是无符号数,左移若干位后,原来最低位右边新出现的位全为0

一个有符号的正数,它的最高位为0,如果因为左移使得最高位为1,那么它变为负数, 而后无论怎样右移,它都还是负数。除非因为再次左移使最高位变为0,那么它变回正数。

int的最大值:\textasciitilde(1<<31),即0后面跟31个1

int的最小值:1<<31,即1后面跟31个0

unsigned int最大值:\textasciitilde 0,即32个1

unsigned int最小值:0

其它数据类型与int类似,只是位长不一样。

完整代码如下:
\lstinputlisting[language=C++]{ch5one.cpp}


\item[5.2] Given a (decimal - e.g. 3.72) number that is passed in as a string, print the binary representation.If the number can not be represented accurately in binary, print “ERROR”.

给定一个字符串类型(string)表示的小数,打印出它的二进制表示。 如果这个数无法精确地表示为二进制形式,输出"ERROR"。

Solution:

整数部分通过不断地对2取余然后除以2来得到其二进制表示, 或是不断地和1按位与然后除以2得到其二进制表示。 小数部分则通过不断地乘以2然后与1比较来得到其二进制表示。 小数部分转化为二进制,通过乘以2然后与1比较,大于等于1则该位为1,并且该值减去1; 否则该位为0。不断地通过这种操作最终能使该小数部分的值变为0的,即可精确表示。 否则将无法用有限的位数来表示这个小数部分。我们可以设定一个长度,比如说32, 在32位之内还无法精确地表示这个小数部分的,我们就认为它无法精确表示了。

代码如下:
\lstinputlisting[language=C++]{ch5two.cpp}


\item[5.3] Given an integer, print the next smallest and next largest number that have the same number of 1 bits in their binary representation.

给定一个整数x,找出另外两个整数,这两个整数的二进制表示中1的个数和x相同, 其中一个是比x大的数中最小的,另一个是比x小的数中最大的。

Solution:

对于这道题,我们先完成一个最朴素最直接的版本,以保证其正确性。 这个版本可以作为其它版本的验证工具。

什么方法是最直接的呢?给定一个数x,计算出它的二进制表示中1的个数为num, 然后x不断加1并计算其二进制表示中1的个数,当它再次等于num时, 就找到了比x大且二进制表示中1的个数相同的最小的数。类似地, 可以找到比x小且二进制表示中1的个数相同的最大的数,只不过x变为不断减1而已。

代码如下:
\begin{lstlisting}[language=C++]
int next(int x){
    int max_int = ~(1<<31);
    int num = count_one(x);
    if(num == 0 || x == -1) return -1;
    for(++x; count_one(x) != num && x < max_int; ++x);
    if(count_one(x) == num) return x;
    return -1;    
}

int previous(int x){
    int min_int = (1<<31);
    int num = count_one(x);
    if(num == 0 || x == -1) return -1;
    for(--x; count_one(x) != num && x > min_int; --x);
    if(count_one(x) == num) return x;
    return -1; 
}
\end{lstlisting}

count\_one函数的功能是计算一个数的二进制表示中1的个数,这个要怎么实现呢? 一种方法是通过不断地移位判断最低位是否为1然后计数器累加,代码如下:
\begin{lstlisting}[language=C++]
int count_one(int x){
    int cnt = 0;
    for(int i=0; i<32; ++i){
        if(x & 1) ++cnt;
        x >>= 1;
    }
    return cnt;
}
\end{lstlisting}

这里for循环可否换成while(x > 0)呢,如果x恒为正数是没问题的,可是如果x为负数, 那么x是无法通过不断地右移变为非负数的。所以这里用for循环比较保险。

这种方法非常直观,不过还有更高效更优美的方法。 这种方法先将相邻的位包含的1的个数相加,然后将相邻每2位包含的1的个数相加, 再然后将相邻每4位包含的1的个数相加……最后即可统计出一个数中包含的1的个数。 代码中大部分操作都是位操作,执行起来非常高效。

代码如下:
\begin{lstlisting}[language=C++]
int count_one(int x){
    x = (x & (0x55555555)) + ((x >> 1) & (0x55555555));
    x = (x & (0x33333333)) + ((x >> 2) & (0x33333333));
    x = (x & (0x0f0f0f0f)) + ((x >> 4) & (0x0f0f0f0f));
    x = (x & (0x00ff00ff)) + ((x >> 8) & (0x00ff00ff));
    x = (x & (0x0000ffff)) + ((x >> 16) & (0x0000ffff));
    return x;
}
\end{lstlisting}

好了,接下来考虑除了朴素方案外,有没有更高效的方案。假设给定的数的二进制表示为: 1101110,我们从低位看起,找到第一个1,从它开始找到第一个0,然后把这个0变为1, 比这个位低的位全置0,得到1110000,这个数比原数大,但比它少了两个1, 直接在低位补上这两个1,得到,1110011,这就是最终答案。 我们可以通过朴素版本来模拟这个答案是怎么得到的:

1101110->1101111->1110000->1110001->1110010->1110011

接下来,我们来考虑一些边界情况,这是最容易被忽略的地方(感谢细心的读者)。 假设一个32位的整数, 它的第31位为1,即:0100..00,那么按照上面的操作,我们会得到1000..00, 很不幸,这是错误的。因为int是有符号的,意味着我们得到了一个负数。 我们想要得到的是一个比0100..00大的数,结果得到一个负数,自然是不对的。 事实上比0100..00大的且1的个数和它一样的整数是不存在的,扩展可知, 对于所有的0111..,都没有比它们大且1的个数和它们一样的整数。对于这种情况, 直接返回-1。-1的所有二进制位全为1,不存在一个数说1的个数和它一样还比它大或小的, 因此适合作为找不到答案时的返回值。

另一个边界情况是什么呢?就是对于形如11100..00的整数,它是一个负数, 比它大且1的个数相同的整数有好多个,最小的当然是把1都放在最低位了:00..0111。

代码如下:
\begin{lstlisting}[language=C++]
int next1(int x){
    int xx = x, bit = 0;
    for(; (x&1) != 1 && bit < 32; x >>= 1, ++bit);
    for(; (x&1) != 0 && bit < 32; x >>= 1, ++bit);
    if(bit == 31) return -1; //011.., none satisify
    x |= 1;
    x <<= bit; // wtf, x<<32 != 0,so use next line to make x=0
    if(bit == 32) x = 0; // for 11100..00
    int num1 = count_one(xx) - count_one(x);
    int c = 1;
    for(; num1 > 0; x |= c, --num1, c <<= 1);
    return x;
}
\end{lstlisting}

类似的方法可以求出另一个数,这里不再赘述。代码如下:
\begin{lstlisting}[language=C++]
int previous1(int x){
    int xx = x, bit = 0;
    for(; (x&1) != 0 && bit < 32; x >>= 1, ++bit);
    for(; (x&1) != 1 && bit < 32; x >>= 1, ++bit);
    if(bit == 31) return -1; //100..11, none satisify
    x -= 1;
    x <<= bit;
    if(bit == 32) x = 0;
    int num1 = count_one(xx) - count_one(x);
    x >>= bit;
    for(; num1 > 0; x = (x<<1) | 1, --num1, --bit);
    for(; bit > 0; x <<= 1, --bit);
    return x;
}
\end{lstlisting}

完整代码如下:
\lstinputlisting[language=C++]{ch5thr.cpp}


\item[5.4] Explain what the following code does: ((n \& (n-1)) == 0).

解答以下代码的作用:((n \& (n-1)) == 0)

Solution:

这个比较简单,代码的作用是判断一个数是否为2的整数次幂。题目中的判断代码不够严谨, 因为当n=0时,上述条件为真,但0并不是2的幂。所以,上述语句可以修改为:

(n > 0) \&\& ((n \& (n-1)) == 0)


\item[5.5] Write a function to determine the number of bits required to convert integer A to integer B.

Input: 31, 14

Output: 2

写程序计算从整数A变为整数B需要修改的二进制位数。

输入:31,14

输出:2

Solution:

这道题目也比较简单,从整数A变到整数B,所需要修改的就只是A和B二进制表示中不同的位, 先将A和B做异或,然后再统计结果的二进制表示中1的个数即可。

代码如下:
\begin{lstlisting}[language=C++]
int count_one(int x){
    x = (x & (0x55555555)) + ((x >> 1) & (0x55555555));
    x = (x & (0x33333333)) + ((x >> 2) & (0x33333333));
    x = (x & (0x0f0f0f0f)) + ((x >> 4) & (0x0f0f0f0f));
    x = (x & (0x00ff00ff)) + ((x >> 8) & (0x00ff00ff));
    x = (x & (0x0000ffff)) + ((x >> 16) & (0x0000ffff));
    return x;
}

int convert_num(int a, int b){
    return count_one(a^b);
}
\end{lstlisting}
完整代码如下:
\lstinputlisting[language=C++]{ch5fiv.cpp}


\item[5.6] Write a program to swap odd and even bits in an integer with as few instructions as possible (e.g., bit 0 and bit 1 are swapped, bit 2 and bit 3 are swapped, etc).

写程序交换一个整数二进制表示中的奇数位和偶数位,用尽可能少的代码实现。 (比如,第0位和第1位交换,第2位和第3位交换…)

Solution:

这道题目比较简单。分别将这个整数的奇数位和偶数位提取出来,然后移位取或即可。

代码如下:
\begin{lstlisting}[language=C++]
int swap_bits(int x){
    return ((x & 0x55555555) << 1) | ((x >> 1) & 0x55555555);
}
\end{lstlisting}

当然也可以采用更自然的方式来写这段代码:
\begin{lstlisting}[language=C++]
int swap_bits1(int x){
    return ((x & 0x55555555) << 1) | ((x & 0xAAAAAAAA) >> 1);
}
\end{lstlisting}

上面的代码思路和作用都是一样的,不过按照《Hacker’s delight》这本书里的说法, 第一种方法避免了在一个寄存器中生成两个大常量。如果计算机没有与非指令, 将导致第二种方法多使用1个指令。总结之,就是第一种方法更好。:P

完整代码如下:
\lstinputlisting[language=C++]{ch5six.cpp}


\item[5.7] An array A[1…n] contains all the integers from 0 to n except for one number which is missing. In this problem, we cannot access an entire integer in A with a single operation. The elements of A are represented in binary, and the only operation we can use to access them is “fetch the jth bit of A[i]”, which takes constant time. Write code to find the missing integer. Can you do it in O(n) time?

数组A[1…n]包含0到n的所有整数,但有一个整数丢失了。在这个问题中, 我们不能直接通过A[i]取得数组中的第i个数。数组A被表示成二进制, 也就是一串的0/1字符,而我们唯一能使用的操作只有“取得A[i]中的第j位”, 这个操作只需要花费常数时间。写程序找出丢失的整数,你能使程序的时间复杂度是O(n)吗?

Solution:

首先,在这个问题中,明确我们唯一能使用的操作是fetch(a, i, j)即: 取得a[i]中的第j位。它是提供给我们的操作,怎么实现的不用去理它, 而我们要利用它来解决这个问题,并且在我们的程序中,不能使用a[i]这样的操作。

如果我们不能直接使用a[i],那么我们能利用fetch函数来获得a[i]吗?回答是可以。 数组a中每个元素是一个整型数,所以只要每次取出32位,再计算出它的值即可。

代码如下:
\begin{lstlisting}[language=C++]
int get(int a[], int i){
    int ret = 0;
    for(int j=31; j>=0; --j)
        ret = (ret << 1) | fetch(a, i, j);
    return ret;
}
\end{lstlisting}

我们已经通过get(a, i)来取得a[i]的值了,这样一来,我们只需要开一个bool数组, 把出现过的整数标记为true,即可找出丢失的那个整数。

代码如下:
\begin{lstlisting}[language=C++]
int missing(int a[], int n){
    bool *b = new bool[n+1];
    memset(b, false, (n+1)*sizeof(bool));
    for(int i=0; i<n; ++i)
        b[get(a, i)] = true;
    for(int i=0; i<n+1; ++i){
        if(!b[i]) return i;
    }
    delete[] b;
}
\end{lstlisting}

我们把问题再变难一点点,如果我们能取到的只是数组a中的第j位,即有函数fetch(a, j), 而不是取到a[i]中的第j位,那又应该怎么做呢。其实也很简单, 计算a[i]需要取出a[i]中的第31位到第0位(共32位),对应到整个数组上, 就是数组从头开始数起,取出第32*i+31位到第32*i位,代码如下:
\begin{lstlisting}[language=C++]
int get1(int a[], int i){
    int ret = 0;
    int base = 32*i;
    for(int j=base+31; j>=base; --j)
        ret = (ret << 1) | fetch1(a, j);
    return ret;
}
\end{lstlisting}
完整代码如下:
\lstinputlisting[language=C++]{ch5sev.cpp}


\end{description}
