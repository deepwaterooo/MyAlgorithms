\chapter{Recursion}
\small{}

\begin{description}
\item[8.1] Write a method to generate the nth Fibonacci number.

写一个函数来产生第n个斐波那契数。

Solution: 

斐波那契数列的定义如下:
\begin{lstlisting}[language=C++]
f(1) = f(2) = 1;
f(n) = f(n-1) + f(n-2);
\end{lstlisting}
这个定义是递归的,因此很容易根据以上的定义写出它的递归解法, 由于这个数列的递增速度飞快XD,我们先重定义一下long long好方便使用:
\begin{lstlisting}[language=C++]
typedef long long ll;
\end{lstlisting}
递归版本:
\begin{lstlisting}[language=C++]
ll fib(ll n){
    if(n < 1) return -1;
    if(n == 1 || n == 2) return 1;
    else return fib(n-1) + fib(n-2);
}
\end{lstlisting}
当然了,根据定义我们也可以很容易地写出它的非递归版本(迭代版本):
\begin{lstlisting}[language=C++]
ll fib1(ll n){
    if(n < 1) return -1;
    if(n == 1 || n == 2) return 1;
    ll a = 1, b = 1;
    for(ll i=3; i<=n; ++i){
        ll c = a + b;
        a = b;
        b = c;
    }
    return b;
}
\end{lstlisting}

空间复杂度O(1),时间复杂度O(n),看起来既简单又快速。可是,我们还有更快的解法。 根据上面的递推公式,我们可以得到它的矩阵版本:

从上图可以看出,写成矩阵递推形式,可以让我们一推到底。最后的f(1)=f(2)=1, 因此,这个问题就转换成了,如何求矩阵的幂。当然了,要快速,不然就没有什么意义了。 我们先把问题退化一下,先不考虑求矩阵的幂,而是求一个整数的幂,这个够简单的吧。

先来看看最naive的解法:(方便起见,这里假设n为非负数,不对n小于0的情况做讨论)
\begin{lstlisting}[language=C++]
ll pow(ll m, ll n){
    ll res = 1;
    for(ll i=0; i<n; ++i)
        res *= m;
    return res;
}
\end{lstlisting}
时间复杂度O(n)。现在让我们来考虑一种更快的方法,假设我们要计算m13 , 然后我们把指数13写成二进制形式13=1101,一开始结果res=1.我们要计算的幂可以写成:

m\^13 = m\^1 * m\^4 * m\^8

我们可以很直观的得出,如果指数13的二进制形式1101中的某一位为1,那么, res就去乘以那一位对应的一个数。比如,1101从低位起,第1位为1,那么res乘以m1 , 第二位为0,res不需要乘以m2 ,第三位为1,res乘以m4 ,第四位为1,res乘以m8 , 最后得到的就是:

res = m\^1 * m\^4 * m\^8

而且由于每次res去乘以的数(如果该位为0则不乘)都是上一次那个数的平方, 所以,这个数我用完一次,就对它取平方,准备下一次的使用即可。看代码:
\begin{lstlisting}[language=C++]
ll pow1(ll m, ll n){
    ll res = 1;
    while(n > 0){
        if(n&1) res *= m;
        m *= m;
        n >>= 1;
    }
    return res;
}
\end{lstlisting}
时间复杂度O(logn),正是我们想要的快速版本。OK, 这时候如果让你快速求矩阵的幂,是不是很简单了?只需要将实数乘法改成矩阵乘法即可。
\begin{lstlisting}[language=C++]
void pow(ll s[2][2], ll a[2][2], ll n){
    while(n > 0){
        if(n&1) mul(s, s, a);
        mul(a, a, a);
        n >>= 1;
    }
}
\end{lstlisting}
基本上是一模一样的,只不过由于计算结果是矩阵,不能直接用return进行返回, 而是在函数的参数列表中返回。矩阵乘法函数如下(只考虑2*2的矩阵乘法)
\begin{lstlisting}[language=C++]
void mul(ll c[2][2], ll a[2][2], ll b[2][2]){
    ll t[4];
    t[0] = a[0][0]*b[0][0] + a[0][1]*b[1][0];
    t[1] = a[0][0]*b[0][1] + a[0][1]*b[1][1];
    t[2] = a[1][0]*b[0][0] + a[1][1]*b[1][0];
    t[3] = a[1][0]*b[0][1] + a[1][1]*b[1][1];
    c[0][0] = t[0];
    c[0][1] = t[1];
    c[1][0] = t[2];
    c[1][1] = t[3];
}
\end{lstlisting}
于是,求斐波那契数列第n项的O(logn)解法如下:
\begin{lstlisting}[language=C++]
ll fib2(ll n){
    if(n < 1) return -1;
    if(n == 1 || n == 2) return 1;

    ll a[2][2] = { {1, 1}, {1, 0} };
    ll s[2][2] = { {1, 0}, {0, 1} };
    pow(s, a, n-2);
    return s[0][0] + s[0][1];
}
\end{lstlisting}
完整代码如下:
\lstinputlisting[language=C++]{ch8one.cpp}


\item[8.2] Imagine a robot sitting on the upper left hand corner of an NxN grid. The robot can only move in two directions: right and down. How many possible paths are there for the robot?

FOLLOW UP

Imagine certain squares are “off limits”, such that the robot can not step on them. Design an algorithm to get all possible paths for the robot.

在一个N*N矩阵的左上角坐着一个机器人,它只能向右运动或向下运动。那么, 机器人运动到右下角一共有多少种可能的路径?

进一步地,

如果对于其中的一些格子,机器人是不能踏上去的。设计一种算法来获得所有可能的路径。

Solution: 

为了一般化这个问题,我们假设这个矩阵是m*n的,左上角的格子是(1, 1), 右下角的坐标是(m, n)。

解法一

这个题目可以递归来解,如何递归呢?首先,我们需要一个递推公式, 对于矩阵中的格子(i, j),假设从(1, 1)到它的路径数量为path(i, j), 那么,有:

path(i, j) = path(i-1, j) + path(i, j-1)

很好理解,因为机器人只能向右或向下运动,因此它只能是从(i-1, j)或(i, j-1) 运动到(i, j)的,所以路径数量也就是到达这两个格子的路径数量之和。然后, 我们需要一个初始条件,也就是递归终止条件,是什么呢?可以发现, 当机器人在第一行时,不论它在第一行哪个位置,从(1, 1)到达那个位置都只有一条路径, 那就是一路向右;同理,当机器人在第一列时,也只有一条路径到达它所在位置。 有了初始条件和递推公式,我们就可以写代码了,如下:
\begin{lstlisting}[language=C++]
ll path(ll m, ll n){
    if(m == 1 || n == 1) return 1;
    else return path(m-1, n) + path(m, n-1);
}
\end{lstlisting}
ll是数据类型long long。

解法二

如果用纯数学的方法来解这道题目,大概也就是个高中排列组合简单题吧。 机器人从(1, 1)走到(m, n)一定要向下走m-1次,向右走n-1次,不管这过程中是怎么走的。 因此,一共可能的路径数量就是从总的步数(m-1+n-1)里取出(m-1)步,作为向下走的步子, 剩余的(n-1)步作为向右走的步子。

C(m-1+n-1, m-1)=(m-1+n-1)! / ( (m-1)! * (n-1)! )

代码如下:
\begin{lstlisting}[language=C++]
ll fact(ll n){
    if(n == 0) return 1;
    else return n*fact(n-1);
}
ll path1(ll m, ll n){
    return fact(m-1+n-1)/(fact(m-1)*fact(n-1));
}
\end{lstlisting}
对于第二问,如果有一些格子,机器人是不能踏上去的(比如说放了地雷XD), 那么,我们如何输出它所有可能的路径呢?

让我们先来考虑简单一点的问题,如果我们只要输出它其中一条可行的路径即可, 那么我们可以从终点(m, n)开始回溯,遇到可走的格子就入栈, 如果没有格子能到达当前格子,当前格子则出栈。最后到达(1, 1)时, 栈中正好保存了一条可行路径。代码如下:
\begin{lstlisting}[language=C++]
bool get_path(int m, int n){
    point p; p.x=n; p.y=m;
    sp.push(p);
    if(n==1 && m==1) return true;
    bool suc = false;
    if(m>1 && g[m-1][n])
        suc = get_path(m-1, n);
    if(!suc && n>1 && g[m][n-1])
        suc = get_path(m, n-1);
    if(!suc) sp.pop();
    return suc;
}
\end{lstlisting}
其中二维数组g表示的是M*N的矩阵,元素为1表示该位置可以走,为0表示该位置不可走。 这个只能得到其中一条可行路径,但题目是要求我们找到所有可行路径,并输出。 这样的话,又该怎么办呢?我们从(1, 1)开始,如果某个格子可以走, 我们就将它保存到路径数组中;如果不能走,则回溯到上一个格子, 继续选择向右或者向下走。当机器人走到右下角的格子(M, N)时,即可输出一条路径。 然后程序会退出递归,回到上一个格子,找寻下一条可行路径。代码如下:
\begin{lstlisting}[language=C++]
void print_paths(int m, int n, int M, int N, int len){
    if(g[m][n] == 0) return;
    point p; p.x=n; p.y=m;
    vp[len++] = p;
    if(m == M && n == N){
        for(int i=0; i<len; ++i)
            cout<<"("<<vp[i].y<<", "<<vp[i].x<<")"<<" ";
        cout<<endl;
    }
    else{
        print_paths(m, n+1, M, N, len);
        print_paths(m+1, n, M, N, len);
    }
}
\end{lstlisting}
程序使用的输入样例8.2.in如下:
\begin{lstlisting}[language=C++]
3 4
1 1 1 0
0 1 1 1
1 1 1 1
\end{lstlisting}
输出路径如下:
\begin{lstlisting}[language=C++]
one of the paths:
(1, 1) (1, 2) (1, 3) (2, 3) (2, 4) (3, 4) 
all paths:
(1, 1) (1, 2) (1, 3) (2, 3) (2, 4) (3, 4) 
(1, 1) (1, 2) (1, 3) (2, 3) (3, 3) (3, 4) 
(1, 1) (1, 2) (2, 2) (2, 3) (2, 4) (3, 4) 
(1, 1) (1, 2) (2, 2) (2, 3) (3, 3) (3, 4) 
(1, 1) (1, 2) (2, 2) (3, 2) (3, 3) (3, 4)
\end{lstlisting}
完整代码如下:
\lstinputlisting[language=C++]{ch8two.cpp}


\item[8.3] Write a method that returns all subsets of a set.

写一个函数返回一个集合中的所有子集。

Solution: 

对于一个集合,它的子集一共有2n 个(包括空集和它本身)。它的任何一个子集, 我们都可以理解为这个集合本身的每个元素是否出现而形成的一个序列。比如说, 对于集合{1, 2, 3},空集表示一个元素都没出现,对应{0, 0, 0}; 子集{1, 3},表示元素2没出现(用0表示),1,3出现了(用1表示),所以它对应 {1, 0, 1}。这样一来,我们发现,{1, 2, 3}的所有子集可以用二进制数000到111 的8个数来指示。泛化一下,如果一个集合有n个元素,那么它可以用0到2n -1 总共2n 个数的二进制形式来指示。每次我们只需要检查某个二进制数的哪一位为1, 就把对应的元素加入到这个子集就OK。代码如下:
\begin{lstlisting}[language=C++]
typedef vector<vector<int> > vvi;
typedef vector<int> vi;
vvi get_subsets(int a[], int n){ //O(n2^n)
    vvi subsets;
    int max = 1<<n;
    for(int i=0; i<max; ++i){
        vi subset;
        int idx = 0;
        int j = i;
        while(j > 0){
            if(j&1){
                subset.push_back(a[idx]);
            }
            j >>= 1;
            ++idx;
        }
        subsets.push_back(subset);
    }
    return subsets;
}
\end{lstlisting}
解这道题目的另一种思路是递归。这道题目为什么可以用递归? 因为我们能找到比原问题规模小却同质的问题。比如我要求{1, 2, 3}的所有子集, 我把元素1拿出来,然后去求{2, 3}的所有子集,{2, 3}的子集同时也是{1, 2, 3} 的子集,然后我们把{2, 3}的所有子集都加上元素1后,又得到同样数量的子集, 它们也是{1, 2, 3}的子集。这样一来,我们就可以通过求{2, 3}的所有子集来求 {1, 2, 3}的所有子集了。而同理,{2, 3}也可以如法炮制。代码如下:
\begin{lstlisting}[language=C++]
vvi get_subsets1(int a[], int idx, int n){
    vvi subsets;
    if(idx == n){
        vi subset;
        subsets.push_back(subset); //empty set
    }
    else{
        vvi rsubsets = get_subsets1(a, idx+1, n);
        int v = a[idx];
        for(int i=0; i<rsubsets.size(); ++i){
            vi subset = rsubsets[i];
            subsets.push_back(subset);
            subset.push_back(v);
            subsets.push_back(subset);
        }
    }
    return subsets;
}
\end{lstlisting}
完整代码如下:
\lstinputlisting[language=C++]{ch8thr.cpp}


\item[8.4] Write a method to compute all permutations of a string

写一个函数返回一个串的所有排列

Solution: 

对于一个长度为n的串,它的全排列共有A(n, n)=n!种。这个问题也是一个递归的问题, 不过我们可以用不同的思路去理解它。为了方便讲解,假设我们要考察的串是"abc", 递归函数名叫permu。

思路一:

我们可以把串“abc”中的第0个字符a取出来,然后递归调用permu计算剩余的串“bc” 的排列,得到{bc, cb}。然后再将字符a插入这两个串中的任何一个空位(插空法), 得到最终所有的排列。比如,a插入串bc的所有(3个)空位,得到{abc,bac,bca}。 递归的终止条件是什么呢?当一个串为空,就无法再取出其中的第0个字符了, 所以此时返回一个空的排列。代码如下:
\begin{lstlisting}[language=C++]
typedef vector<string> vs;

vs permu(string s){
    vs result;
    if(s == ""){
        result.push_back("");
        return result;
    }
    string c = s.substr(0, 1);
    vs res = permu(s.substr(1));
    for(int i=0; i<res.size(); ++i){
        string t = res[i];
        for(int j=0; j<=t.length(); ++j){
            string u = t;
            u.insert(j, c);
            result.push_back(u);
        }
    }
    return result; //`调用result的拷贝构造函数,返回它的一份copy,然后这个局部变量销毁(与基本类型一样)`
}
\end{lstlisting}
思路二:

我们还可以用另一种思路来递归解这个问题。还是针对串“abc”, 我依次取出这个串中的每个字符,然后调用permu去计算剩余串的排列。 然后只需要把取出的字符加到剩余串排列的每个字符前即可。对于这个例子, 程序先取出a,然后计算剩余串的排列得到{bc,cb},然后把a加到它们的前面,得到 {abc,acb};接着取出b,计算剩余串的排列得到{ac,ca},然后把b加到它们前面, 得到{bac,bca};后面的同理。最后就可以得到“abc”的全序列。代码如下:
\begin{lstlisting}[language=C++]
vs permu1(string s){
    vs result;
    if(s == ""){
        result.push_back("");
        return result;
    }
    for(int i=0; i<s.length(); ++i){
        string c = s.substr(i, 1);
        string t = s;
        vs res = permu1(t.erase(i, 1));
        for(int j=0; j<res.size(); ++j){
            result.push_back(c + res[j]);
        }
    }
    return result;
}
\end{lstlisting}
完整代码如下:
\lstinputlisting[language=C++]{ch8for.cpp}


\item[8.5] Implement an algorithm to print all valid (e.g., properly opened and closed) combinations of n-pairs of parentheses.

EXAMPLE:

input: 3 (e.g., 3 pairs of parentheses)

output: ((())), (()()), (())(), ()(()), ()()()

实现一个算法打印出n对括号的有效组合。

输入:3 (3对括号)

输出:((())), (()()), (())(), ()(()), ()()()

Solution: 

对于括号的组合,要考虑其有效性。比如说,)(, 它虽然也是由一个左括号和一个右括号组成,但它就不是一个有效的括号组合。 那么,怎样的组合是有效的呢?对于一个左括号,在它右边一定要有一个右括号与之配对, 这样的才能是有效的。所以,对于一个输出,比如说(()()), 从左边起,取到任意的某个位置得到的串,左括号数量一定是大于或等于右括号的数量, 只有在这种情况下,这组输出才是有效的。我们分别记左,右括号的数量为left和right, 如下分析可看出,(()())是个有效的括号组合。
\begin{lstlisting}[language=C++]
(, left = 1, right = 0
((, left = 2, right = 0
((), left = 2, right = 1
(()(, left = 3, right = 1
(()(), left = 3, right = 2
(()()), left = 3, right = 3
\end{lstlisting}
这样一来,在程序中,只要还有左括号,我们就加入输出串,然后递归调用。 当退出递归时,如果剩余的右括号数量比剩余的左括号数量多,我们就将右括号加入输出串。 直到最后剩余的左括号和右括号都为0时,即可打印一个输出串。代码如下:
\begin{lstlisting}[language=C++]
void print_pare(int l, int r, char str[], int cnt){
    if(l<0 || r<l) return;
    if(l==0 && r==0){
        for(int i=0; i<cnt; ++i){
            cout<<str[i];
        }
        cout<<", ";
    }
    else{
        if(l > 0){
            str[cnt] = '(';
            print_pare(l-1, r, str, cnt+1);
        }
        if(r > l){
            str[cnt] = ')';
            print_pare(l, r-1, str, cnt+1);
        }
    }
}
\end{lstlisting}
完整代码如下:
\lstinputlisting[language=C++]{ch8fiv.cpp}


\item[8.6] Implement the “paint fill” function that one might see on many image editing programs. That is, given a screen (represented by a 2-dimensional array of Colors), a point, and a new color, fill in the surrounding area until you hit a border of that color.

实现图像处理软件中的“填充”函数,给定一块区域(可以不规则),一个种子点和一种新颜色, 填充这块区域,直到到达这块区域的边界(边界是用这种新颜色画的一条封闭曲线)

Solution: 

两种思路,一种递归,一种迭代。基本上能递归的都能迭代, 只不过有时迭代版本不是太好写。如果我没有记错, 填充函数在计算机图像学的书里是会讲的。给你一个种子点和一个颜色, 从这个种子点开始去把这块区域都填充上这种颜色。我们可以从这个种子点开始着色, 然后递归调用本函数去给它的上,下,左,右四个点着色。 递归停止条件是碰到这个区域的边界(边界是一条同种颜色绘制的封闭曲线)。比如说, 我先用绿色画了一个圆形(或是其它任意形状),然后在这个形状中间再用绿色点击一下, 那么这块区域就被填充成绿色了。下面是代码:
\begin{lstlisting}[language=C++]
bool paint_fill(color **screen, int m, int n, int x, int y, color c){
    if(x<0 || x>=m || y<0 || y>=n) return false;
    if(screen[x][y] == c) return false;
    else{
        screen[x][y] = c;
        paint_fill(screen, m, n, x-1, y, c);
        paint_fill(screen, m, n, x+1, y, c);
        paint_fill(screen, m, n, x, y-1, c);
        paint_fill(screen, m, n, x, y+1, c);
    }
    return true;
}
\end{lstlisting}
迭代版本也不难,用BFS即可。每次给一个点着色后,就把它周围四个点放入队列, 只要队列不为空,就一直处理。这里需要注意一点的是,如果遇到一个点已经是新颜色, (比如上面说的绿色),那么我就不用处理它而且不需要把它周围四个点入队。 如果不注意这点,那么填充色就会不顾边界直接把整张图片都填充成同一种颜色。 迭代版本的就不写了。以下是完整代码:
\lstinputlisting[language=C++]{ch8six.cpp}


\item[8.7] Given an infinite number of quarters (25 cents), dimes (10 cents), nickels (5 cents) and pennies (1 cent), write code to calculate the number of ways of representing n cents.

我们有25分,10分,5分和1分的硬币无限个。写一个函数计算组成n分的方式有几种?

Solution: 

一开始,我觉得使用递归不断地累加四种币值的硬币,当累加到n分,组成方式就加1。 最后就能得到一共有多少种组合方式,听起来挺正确的,然后写了以下代码:
\begin{lstlisting}[language=C++]
int cnt = 0;
void sumn(int sum, int n){
    if(sum >= n){
        if(sum == n) ++cnt;
        return;
    }
    else{
        sumn(sum+25, n);
        sumn(sum+10, n);
        sumn(sum+5, n);
        sumn(sum+1, n);
    }
}
\end{lstlisting}
但,这是错误的。问题出在哪?有序与无序的区别!这个函数计算出来的组合是有序的, 也就是它会认为1,5和5,1是不一样的,导致计算出的组合里有大量是重复的。 那要怎么避免这个问题?1,5和5,1虽然会被视为不一样,但如果它们是排好序的, 比如都按从大到小排序,那么就都是5,1了,这时就不会重复累计组合数量。 可是我们总不能求出答案来后再搞个排序吧,多费劲。这时我们就可以在递归上做个手脚, 让它在计算的过程中就按照从大到小的币值来组合。比如, 现在我拿了一个25分的硬币,那下一次可以取的币值就是25,10,5,1;如果我拿了一个 10分的,下一次可以取的币值就只有10,5,1了;这样一来,就能保证,同样的组合, 我只累计了一次,改造后的代码如下:
\begin{lstlisting}[language=C++]
int cnt = 0;
void sumN(int sum, int c, int n){
    if(sum >= n){
        if(sum == n) ++cnt;
        return;
    }
    else{
        if(c >= 25)
            sumN(sum+25, 25, n);
        if(c >= 10)
            sumN(sum+10, 10, n);
        if(c >= 5)
            sumN(sum+5, 5, n);
        if(c >= 1)
            sumN(sum+1, 1, n);
    }
}
\end{lstlisting}
有个全局变量总觉得看起来不好看,于是我们可以把这个全局变量去掉, 让函数来返回这个组合数目,代码如下:
\begin{lstlisting}[language=C++]
int sum_n(int sum, int c, int n){
    int ways = 0;
    if(sum <= n){
        if(sum == n) return 1;
        if(c >= 25)
            ways += sum_n(sum+25, 25, n);
        if(c >= 10)
            ways += sum_n(sum+10, 10, n);
        if(c >= 5)
            ways += sum_n(sum+5, 5, n);
        if(c >= 1)
            ways += sum_n(sum+1, 1, n);
    }
    return ways;
}
\end{lstlisting}
CTCI原文中给出的解法如下:
\begin{lstlisting}[language=C++]
int make_change(int n, int denom){
    int next_denom = 0;
    switch(denom){
    case 25:
        next_denom = 10;
        break;
    case 10:
        next_denom = 5;
        break;
    case 5:
        next_denom = 1;
        break;
    case 1:
        return 1;
    }
    int ways = 0;
    for(int i=0; i*denom<=n; ++i)
        ways += make_change(n-i*denom, next_denom);
    return ways;
}
\end{lstlisting}
也是从币值大的硬币开始,每次取i个(i乘以币值要小于等于n), 然后接着去取币值比它小的硬币,这时就是它的一个子问题了,递归调用。 具体来怎么来理解这个事呢?这样,比如我要凑100分,那我先从25分开始, 我先取0个25分,然后递归调用去凑100分;再然后我取1个25分,然后递归调用去凑100-25 =75分;接着我取2个25分,递归调用去凑100-25*2=50分……这些取了若干个 25分然后再去递归调用,取的就是10分了。一直这样操作下去,我们就会得到, 由若干个25,若干个10分,若干个5分和若干个1分组成的100分,而且, 这里面每种币值的数量都可以为0。

完整代码如下:
\lstinputlisting[language=C++]{ch8sev.cpp}


\item[8.8] Write an algorithm to print all ways of arranging eight queens on a chess board so that none of them share the same row, column or diagonal.

经典的八皇后问题,即在一个8*8的棋盘上放8个皇后,使得这8个皇后无法互相攻击( 任意2个皇后不能处于同一行,同一列或是对角线上),输出所有可能的摆放情况。

Solution: 

8皇后是个经典的问题,如果使用暴力法,每个格子都去考虑放皇后与否,一共有264 种可能。所以暴力法并不是个好办法。由于皇后们是不能放在同一行的, 所以我们可以去掉“行”这个因素,即我第1次考虑把皇后放在第1行的某个位置, 第2次放的时候就不用去放在第一行了,因为这样放皇后间是可以互相攻击的。 第2次我就考虑把皇后放在第2行的某个位置,第3次我考虑把皇后放在第3行的某个位置, 这样依次去递归。每计算1行,递归一次,每次递归里面考虑8列, 即对每一行皇后有8个可能的位置可以放。找到一个与前面行的皇后都不会互相攻击的位置, 然后再递归进入下一行。找到一组可行解即可输出,然后程序回溯去找下一组可靠解。

我们用一个一维数组来表示相应行对应的列,比如c[i]=j表示, 第i行的皇后放在第j列。如果当前行是r,皇后放在哪一列呢?c[r]列。 一共有8列,所以我们要让c[r]依次取第0列,第1列,第2列……一直到第7列, 每取一次我们就去考虑,皇后放的位置会不会和前面已经放了的皇后有冲突。 怎样是有冲突呢?同行,同列,对角线。由于已经不会同行了,所以不用考虑这一点。 同列:c[r]==c[j]; 同对角线有两种可能,即主对角线方向和副对角线方向。 主对角线方向满足,行之差等于列之差:r-j==c[r]-c[j]; 副对角线方向满足, 行之差等于列之差的相反数:r-j==c[j]-c[r]。 只有满足了当前皇后和前面所有的皇后都不会互相攻击的时候,才能进入下一级递归。
\lstinputlisting[language=C++]{ch8eit.cpp}

\end{description}
