\chapter{Threads and Locks}

\begin{description}
\item[18.1] What’s the difference between a thread and a process?

线程和进程的区别是什么?

Solution: 

这是一道出现频率极高的面试题,考察基本概念。

进程可以认为是程序执行时的一个实例。进程是系统进行资源分配的独立实体, 且每个进程拥有独立的地址空间。一个进程无法直接访问另一个进程的变量和数据结构, 如果希望让一个进程访问另一个进程的资源,需要使用进程间通信,比如:管道,文件, 套接字等。

一个进程可以拥有多个线程,每个线程使用其所属进程的栈空间。 线程与进程的一个主要区别是,同一进程内的多个线程会共享部分状态, 多个线程可以读写同一块内存(一个进程无法直接访问另一进程的内存)。同时, 每个线程还拥有自己的寄存器和栈,其它线程可以读写这些栈内存。

线程是进程的一个特定执行路径。当一个线程修改了进程中的资源, 它的兄弟线程可以立即看到这种变化。

以下是分点小结:
\begin{enumerate}
\item 进程是系统进行资源分配的基本单位,有独立的内存地址空间; 线程是CPU调度的基本单位,没有单独地址空间,有独立的栈,局部变量,寄存器, 程序计数器等。
\item 创建进程的开销大,包括创建虚拟地址空间等需要大量系统资源; 创建线程开销小,基本上只有一个内核对象和一个堆栈。
\item 一个进程无法直接访问另一个进程的资源;同一进程内的多个线程共享进程的资源。
\item 进程切换开销大,线程切换开销小;进程间通信开销大,线程间通信开销小。
\item 线程属于进程,不能独立执行。每个进程至少要有一个线程,成为主线程
\end{enumerate}

\item[18.2] How can you measure the time spent in a context switch?

你如何测量一次上下文切换所需时间?

Solution: 

这是一个棘手的问题,让我们先从可能的解答入手。

上下文切换(有时也称为进程切换或任务切换)是指CPU 的控制权从一个进程或线程切换到另一个。 (参考资料) 例如让一个正在执行的进程进入等待状态(或终止它),同时去执行另一个正在等待的进程。 上下文切换一般发生在多任务系统中,操作系统必须把等待进程的状态信息载入内存, 同时保存正在运行的进程的状态信息(因为它马上就要变成等待状态了)。

为了解决这个问题,我们需要记录两个进程切换时第一条指令和最后一条指令的时间戳, 上下文切换就是这两个时间戳的差。

来看一个简单的例子,假设只有两个进程:P1和P2。

P1正在执行,而P2在等待。在某个时刻,操作系统从P1切换到P2。假设此时, P1执行到第N条指令,记录时间戳Time\_Stamp(P1\_N),当本来在等待的P2 开始执行第1条指令,说明切换完成,记录时间戳Time\_Stamp(P2\_1)。因此, 上下文切换的时间为:Time\_Stamp(P2\_1) - Time\_Stamp(P1\_N)

思路非常简单。问题在于,我们如何知道上下文切换是何时发生的? 进程的切换是由操作系统的调度算法决定的。我们也无法记录进程中每个指令的时间戳。

另一个问题是:许多内核级线程也做上下文切换,而用户对于它们是没有任何控制权限的。

总而言之,我们可以认为,这最多只能是依赖于底层操作系统的近似计算。 一个近似的解法是记录一个进程结束时的时间戳,另一个进程开始的时间戳及排除等待时间。

如果所有进程总共用时为T,那么总的上下文切换时间为: T - (所有进程的等待时间和执行时间)


\item[18.3] Implement a singleton design pattern as a template such that, for any given class Foo, you can call Singleton::instance() and get a pointer to an instance of a singleton of type Foo. Assume the existence of a class Lock which has acquire() and release() methods. How could you make your implementation thread safe and exception safe?

实现一个单例模式的模板,当给一个类Foo时,你可以通过Singleton::instance() 来得到一个指向Foo类单例的指针。假设我们现在已经有了Lock类,其中有acquire() 和release()两个方法,你要如何使你的实现线程安全且异常安全?

Solution: 
\begin{lstlisting}[language=C++]
#include <iostream>
using namespace std;

/* `线程同步锁` */
class Lock {
public:
    Lock() {  /* `构造锁` */ }
    ~Lock() { /* `析构锁` */ }
    void AcquireLock() { /* `加锁操作` */ }
    void ReleaseLock() { /* `解锁操作` */ }
};

// `单例模式模板,只实例化一次`
template <typename T>
class Singleton{
private:
    static Lock lock;
    static T* object;
protected:
    Singleton() { };
public:
    static T* Instance();
};

template <typename T>
Lock Singleton<T>::lock;

template <typename T>
T* Singleton<T>::object = NULL;

template <typename T>
T* Singleton<T>::Instance(){
    if (object == NULL){// `如果object未初始化,加锁初始化`
        lock.AcquireLock();
        //`这里再判断一次,因为多个线程可能同时通过第一个if`
        //`只有第一个线程去实例化object,之后object非NULL`
        //`后面的线程不再实例化它`
        if (object == NULL){
            object = new T;
        }
        lock.ReleaseLock();
    }
    return object;
}
class Foo{
    
};
int main(){
    Foo* singleton_foo = Singleton<Foo>::Instance();
    return 0;
}
\end{lstlisting}
使一个程序线程安全的一般方法是,当线程获得对共享资源的写权限时,需要对共享资源加锁。 这样一来,当一个线程在修改资源时,另外的线程就不能修改它。

关于这个问题,可以参考《剑指offer》中的面试题2:实现Singleton模式。 那里有详细的解答。
%\lstinputlisting[language=C++]{ch18thr.cpp}


\item[18.4] Design a class which provides a lock only if there are no possible deadlocks.
%\lstinputlisting[language=C++]{ch18for.cpp}


\item[18.5] Suppose we have the following code:
\begin{lstlisting}[language=C++]
class Foo{
public:
    A(.....); /*If A is called, a new thread will be created and
               the corresponding function will be executed. */
    B(.....); /*same as above */
    C(.....); /*same as above */
};
Foo f;
f.A(.....);
f.B(.....);
f.C(.....);
\end{lstlisting}
\begin{description}
\item[i)] Can you design a mechanism to make sure that B is executed after A, and C is executed after B?
\item[ii)] Suppose we have the following code to use class Foo. We do not know how the threads will be scheduled in the OS.
\end{description}
\begin{lstlisting}[language=C++]
Foo f;
f.A(.....); f.B(.....); f.C(.....); 
f.A(.....); f.B(.....); f.C(.....);
\end{lstlisting}
Can you design a mechanism to make sure that all the methods will be executed in sequence?

假设我们有以下代码:

\begin{lstlisting}[language=C++]
class Foo{
public:
    A(.....); /*`当A被调用时,会创建一个新的线程并执行相应的函数`*/
    B(.....); /*`同上`*/
    C(.....); /*`同上`*/
};
Foo f;
f.A(.....);
f.B(.....);
f.C(.....);
\end{lstlisting}
\begin{description}
\item[i)] 你能设计一种机制确保B在A后执行,C在B后执行吗?
\item[ii)] 假设我们有以下代码,我们并不知道操作系统如何调度线程。 你能设计一种机制来确保所有的方法都按顺序执行吗?
\end{description}
\begin{lstlisting}[language=C++]
Foo f;
f.A(.....); f.B(.....); f.C(.....); 
f.A(.....); f.B(.....); f.C(.....);
\end{lstlisting}

Solution: 

第一问,初始的两个信号量都为0,函数A执行完后,信号量s\_a会加1,这时B才可执行。 B执行完后信号量s\_b加1,这时C才可执行。以此保证A,B,C的执行顺序。 注意到函数A其实没有受到限制,所以A可以被多个线程多次执行。比如A执行3次, 此时s\_a=3;然后执行B,s\_a=2,s\_b=1;然后执行C,s\_a=2,s\_b=0; 然后执行B,s\_a=1,s\_b=1。即可以出现类似这种序列:AAABCB。
\begin{lstlisting}[language=C++]
Semaphore s_a(0);
Semaphore s_b(0);
A {
    s_a.release(1); // `信号量s\_a加1`
}
B {
    s_a.acquire(1); // `信号量s\_a减1`
    s_b.release(1); // `信号量s\_b加1`
}
C {
    s_b.acquire(1); // `信号量s\_b减1`
}
\end{lstlisting}
第二问代码如下,与第一问不同,以下代码可以确保执行顺序一定严格按照: ABCABCABC…进行。因为每一时刻都只有一个信号量不为0, 且B中要获取的信号量在A中释放,C中要获取的信号量在B中释放, A中要获取的信号量在C中释放。这个保证了执行顺序一定是ABC。
\begin{lstlisting}[language=C++]
Semaphore s_a(0);
Semaphore s_b(0);
Semaphore s_c(1);
A {
    s_c.acquire(1);
    /***/
    s_a.release(1);
}
B {
    s_a.acquire(1);
    /****/
    s_b.release(1);
}
C {
    s_b.acquire(1);
    /******/
    s_c.release(1);
}
\end{lstlisting}

%\lstinputlisting[language=C++]{ch18fiv.cpp}


\item[18.6] You are given a class with synchronized method A, and a normal method C. If you have two threads in one instance of a program, can they call A at the same time? Can they call A and C at the same time?
%\lstinputlisting[language=C++]{ch18six.cpp}

\end{description}
