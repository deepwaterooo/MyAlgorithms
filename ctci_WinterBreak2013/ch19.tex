\chapter{Moderate}

\begin{description}
\item[19.1] Write a function to swap a number in place without temporary variables.

写一个函数交换两个数,不能使用临时变量。

Solution: 

交换函数swap是经常用到的函数,小巧简单,以下两种实现方式都不需要使用临时变量:
\begin{lstlisting}[language=C++]
// `实现1`
void swap(int &a, int &b){
    b = a - b;
    a = a - b;
    b = a + b;
}
// `实现2`
void swap(int &a, int &b){
    a = a ^ b;
    b = a ^ b;
    a = a ^ b;
}
\end{lstlisting}
以上的swap函数,尤其是第2个实现,简洁美观高效,乃居家旅行必备良品。但是, 使用它们之前一定要想一想,你的程序中,是否有可能会让swap中的两个形参引用同一变量。 如果是,那么上述两个swap函数都将出问题。有人说,谁那么无聊去swap同一个变量。 那可不好说,比如你在操作一个数组中的元素,然后用到了以下语句:
\begin{lstlisting}[language=C++]
swap(a[i], a[j]); // i==j时,出问题
\end{lstlisting}
你并没有注意到swap会去操作同一变量,可是当i等于j时,就相当于你这么干了。 然后呢,上面两个实现执行完第一条语句后,操作的那个内存中的数就变成0了。 后面的语句不会起到什么实际作用。

所以如果程序中有可能让swap函数去操作同一变量,就老老实实用最朴素的版本:
\begin{lstlisting}[language=C++]
void swap(int &a, int &b){
    int t = a;
    a = b;
    b = t;
}
\end{lstlisting}
%\lstinputlisting[language=C++]{ch19one.cpp}


\item[19.2] Design an algorithm to figure out if someone has won in a game of tic-tac-toe.

设计算法检查某人是否赢得了井字游戏。

Solution: 

假设这个检查某人是否赢得了井字游戏的函数为HasWon,那么我们第一步要向面试官确认, 这个函数是只调用一次,还是要多次频繁调用。如果是多次调用, 我们可以通过预处理来得到一个非常快速的版本。
\begin{description}
\item[方法一:] 如果HasWon函数需要被频繁调用

对于井字游戏,每个格子可以是空,我的棋子和对方的棋子3种可能,总共有39 = 19683 种可能状态。我们可以把每一种状态转换成一个整数, 预处理时把这个状态下是否有人赢得了井字游戏保存起来,每次检索时就只需要O(1)时间。 比如每个格子上的3种状态为0(空),1(我方棋子),2(对方棋子),棋盘从左到右, 从上到下依次记为0到8,那么任何一个状态都可以写成一个3进制的数,再转成10进制即可。 比如,下面的状态:
\begin{itemize}
  \itemsep=-3pt
\item 1 2 2
\item 2 1 1
\item 2 0 1
\item 可以写成:
\item 122211201=1*3\^8 + 2*3\^7 +...+ 0*3\^1 + 1*3\^0
\end{itemize}
如果只需要返回是否有人赢,而不需要知道是我方还是对方。 那可以用一个二进制位来表示是否有人赢。比如上面的状态,1赢, 就可以把那个状态转换成的数对应的位置1。如果需要知道是谁赢, 可以用一个char数组来保存预处理结果。

\item[方法二:] 如果HasWon函数只被调用一次或很少次

如果HasWon函数只被调用一次或很少次,那我们就没必要去预存结果了, 直接判断一下就好。只为一次调用去做预处理是不值得的。

代码如下,判断n*n的棋盘是否有人赢,即同一棋子连成一线:
\end{description}
\begin{lstlisting}[language=C++]
#include <iostream>
using namespace std;

enum Check {ROW, COLUMN, DIAGONAL, REDIAGONAL};

int CheckRowColumn(int board[], int n, Check check) {
    int type = 0;
    for (int i = 0; i < n; ++i) {
	bool found = true; 
	for (int j = 0; j < n; ++j) {
	    int k = 0;
	    if (check == ROW)
		k = i * n + j;
	    else 
		k = i + j * n;

	    if (j == 0)
		type = board[k];
	    else if ( board[k] != type ) {
		found = false;
		break; //`有一个不满足,检查下一行`
	    }
	}
	if (found) return type;
    }
    return 0;
}

int CheckDiagonal(int board[], int n, Check check) {
    int type = 0;
    bool found = true;
    // `主对角`
    for (int i = 0; i < n; ++i) {
	int k = 0;
	if (check == DIAGONAL)
	    k = i + i * n;
	else 
	    k = i + (n-1-i) * n;
	if (i == 0) 
	    type = board[k];
	else if (board[k] != type) {
	    found = false;
	    break;
	}
    }
    if (found) return type;
    return 0;
}

int HasWon(int board[], int n) {
    int type = 0;
    type = CheckRowColumn(board, n, ROW);
    if (type != 0) return type;
    type = CheckRowColumn(board, n, COLUMN);
    if (type != 0) return type;
    type = CheckDiagonal(board, n, DIAGONAL);
    if (type != 0) return type;
    type = CheckDiagonal(board, n, REDIAGONAL);
    if (type != 0) return type;
    return 0;
}

int main() {
    int n = 3; // 3 * 3
    int board[] = {2, 2, 1, 2, 1, 1, 1, 2, 0};
    int type = HasWon(board, n);
    if (type != 0)
	cout << type << " won!" << endl;
    else 
	cout << "Nobody won!" << endl;
    return 0;
}
\end{lstlisting}
%\lstinputlisting[language=C++]{ch19two.cpp}


\item[19.3] Write an algorithm which computes the number of trailing zeros in n factorial.

写一个算法计算n的阶乘末尾0的个数。

Solution: 

首先,算出n的阶乘的结果再去计算末尾有多少个0这种方法是不可取的, 因为n的阶乘是一个非常大的数,分分种就会溢出。我们应当去分析, 是什么使n的阶乘结果末尾出现0。

n阶乘末尾的0来自因子5和2相乘,5*2=10。因此,我们只需要计算n的阶乘里, 有多少对5和2。注意到2出现的频率比5多,因此,我们只需要计算有多少个因子5即可。 我们可以列举一些例子,看看需要注意些什么:
\begin{itemize}
  \itemsep=-3pt
\item  5!, 包含1*5, 1个5
\item 10!, 包含1*5,2*5, 2个5
\item 15!, 包含1*5,2*5,3*5, 3个5
\item 20!, 包含1*5,2*5,3*5,4*5, 4个5
\item 25!, 包含1*5,2*5,3*5,4*5,5*5, 6个5
\item ...
\end{itemize}
给定一个n,用n除以5,得到的是从1到n中包含1个5的数的个数;然后用n除以5去更新n, 相当于把每一个包含5的数中的因子5取出来一个。然后继续同样的操作,让n除以5, 将得到此时仍包含有5的数的个数,依次类推。最后把计算出来的个数相加即可。 比如计算25的阶乘中末尾有几个0, 先用25除以5得到5,表示我们从5,10,15,20,25中各拿一个因子5出来,总共拿了5个。 更新n=25/5=5,再用n除以5得到1,表示我们从25中拿出另一个因子5, 其它的在第一次除以5后就不再包含因子5了。

代码如下:
\begin{lstlisting}[language=C++]
int NumZeros(int n){
    if(n < 0) return -1;
    int num = 0;
    while((n /= 5) > 0){
        num += n;
    }
    return num;
}
\end{lstlisting}

%\lstinputlisting[language=C++]{ch19thr.cpp}


\item[19.4] Write a method which finds the maximum of two numbers. You should not use if-else or any other comparison operator.

Input: 5, 10

Output: 10

写一个函数返回两个数中的较大者,你不能使用if-else及任何比较操作符。

Solution: 

我们可以通过一步步的分析来将需要用到的if-else和比较操作符去掉:
\begin{itemize}
  \itemsep=-3pt
\item If a > b, return a; else, return b.
\item If (a - b) < 0, return b; else, return a.
\item If (a - b) < 0, 令k = 1; else, 令k = 0. return a - k * (a - b).
\item 令z = a - b. 令k是z的最高位,return a - k * z.
\end{itemize}
当a大于b的时候,a-b为正数,最高位为0,返回的a-k*z = a;当a小于b的时候, a-b为负数,最高位为1,返回的a-k*z = b。可以正确返回两数中较大的。

另外,k是z的最高位(0或1),我们也可以用一个数组c来存a,b,然后返回c[k]即可。

代码如下:
\lstinputlisting[language=C++]{ch19for.cpp}


\item[19.5] The Game of Master Mind is played as follows:

The computer has four slots containing balls that are red (R ), yellow (Y), green (G) or blue (B). For example, the computer might have RGGB (e.g., Slot \#1 is red, Slots \#2 and \#3 are green, Slot \#4 is blue).

You, the user, are trying to guess the solution. You might, for example, guess YRGB.When you guess the correct color for the correct slot, you get a “hit”. If you guess a color that exists but is in the wrong slot, you get a “pseudo-hit”. For example, the guess YRGB has 2 hits and one pseudo hit.

For each guess, you are told the number of hits and pseudo-hits. Write a method that, given a guess and a solution, returns the number of hits and pseudo hits.

Master Mind游戏规则如下:

4个槽,里面放4个球,球的颜色有4种,红(R ),黄(Y),绿(G),蓝(B)。比如, 给出一个排列RGGB,表示第一个槽放红色球,第二和第三个槽放绿色球,第四个槽放蓝色球。

你要去猜这个排列。比如你可能猜排列是:YRGB。当你猜的颜色是正确的,位置也是正确的, 你就得到一个hit,比如上面第3和第4个槽猜的和真实排列一样(都是GB),所以得到2个hit。 如果你猜的颜色在真实排列中是存在的,但位置没猜对,你就得到一个pseudo-hit。比如, 上面的R,猜对了颜色,但位置没对,得到一个pseudo-hit。

对于你的每次猜测,你会得到两个数:hits和pseudo-hits。写一个函数, 输入一个真实排列和一个猜测,返回hits和pseudo-hits。

Solution: 

这个问题十分直观,但有一个地方需要去向面试官明确一下题意。关于pseudo-hits的定义, 猜对颜色但位置没对,得到一个pseudo-hit,这里是否可以包含重复?举个例子, 比如真实排列是:RYGB,猜测是:YRRR。那么hits很明显为0了。pseudo-hits呢? 猜测中Y对应真实排列中的Y,得到一个pseudo-hits。猜测中有3个R, 而真实排列中只有一个,那这里应该认为得到1个pseudo-hits还是3个? CTCI书认为是3个,想必理由是猜测中的3个R都满足:出现在真实排列中,位置不正确。 所以算3个。但我认为,这里算1个比较合理,真实排列中的R只和猜测中的一个R配对, 剩余的没有配对,不算。弄清题意后,代码就不难写出了。

以下是两种不同理解的实现:
\lstinputlisting[language=C++]{ch19fiv.cpp}


\item[19.6] Given an integer between 0 and 999,999, print an English phrase that describes the integer (eg, “One Thousand, Two Hundred and Thirty Four”).	
%\lstinputlisting[language=C++]{ch19six.cpp}


\item[19.7] You are given an array of integers (both positive and negative). Find the continuous sequence with the largest sum. Return the sum.

EXAMPLE

Input: {2, -8, 3, -2, 4, -10}

Output: 5 (i.e., {3, -2, 4} )

给出一个整数数组(包含正数和负数),找到和最大的连续子序列,返回和。

输入: {2, -8, 3, -2, 4, -10}

输出: 5 (即, {3, -2, 4} )

Solution: 

经典的面试题目,遍历一遍数组,用变量maxsum保存遍历过程中的最大和, 用变量cursum保存遍历过程中的当前和。在遍历的过程中,我们只需要做3件事, 第一:如果当前和cursum小于等于0,说明前面的连续和不会对后面的连续和产生贡献, 要么使后面的连续和减少,要么不变。因此舍弃cursum,用当前的元素更新它。 第二:如果当前和cursum是大于0的,累加当前元素。第三:如果当前和cursum 大于最大和maxsum,则更新最大和maxsum。

此外,我们可以用一个全局变量来标示非法输入。代码如下:
\lstinputlisting[language=C++]{ch19sev.cpp}


\item[19.8] Design a method to find the frequency of occurrences of any given word in a book.

设计一个函数,找到给定单词在一本书中的出现次数。

Solution: 

这道题目和19.2是一个思路。如果只需要查询一次, 那就直接做;如果要多次查询,而且要查询各种不同的单词,那就先预处理一遍, 接下来就只需要用O(1)的时间进行查询。
\begin{description}
\item[只查询一次] 遍历这本书的每个单词,计算给定单词出现的次数。时间复杂度O(n),我们无法继续优化它, 因为书中每个单词都需要访问一次。当然,如果我们假设书中的单词是均匀分布, 那我们就可以只统计前半本书某个单词出现的次数,然后乘以2; 或是只统计前四分之一本书某个单词出现的次数,然后乘以4。这样能计算出一个大概值。 当然,单词均匀分布这个假设太强了,一般是不成立的。
\item[多次查询] 如果我们要对一本书进行多次的查询,就可以遍历一次这本书的单词, 把它出现的次数存入哈希表中。查询的时候即可用O(1)的时间完成。
\end{description}
%\lstinputlisting[language=C++]{ch19eit.cpp}


\item[19.9] Since XML is very verbose, you are given a way of encoding it where each tag gets mapped to a pre-defined integer value. The language/grammar is as follows:	
\begin{itemize}	
\item Element --> Element Attr* END Element END [aka, encode the element tag, then its attributes, then tack on an END character, then encode its children, then another end tag]
\item Attr --> Tag Value [assume all values are strings]
\item END --> 01
\item Tag --> some predefined mapping to int
\item Value --> string value END
\end{itemize}
Write code to print the encoded version of an xml element (passed in as string).

FOLLOW UP

Is there anything else you could do to (in many cases) compress this even further?
%\lstinputlisting[language=C++]{ch19nin.cpp}


\item[19.10] Write a method to generate a random number between 1 and 7, given a method that generates a random number between 1 and 5 (i.e., implement rand7() using rand5()).

给你一个能生成1到5随机数的函数,用它写一个函数生成1到7的随机数。 (即,使用函数rand5()来实现函数rand7())。

Solution: 

rand5可以随机生成1,2,3,4,5;rand7可以随机生成1,2,3,4,5,6,7。 rand5并不能直接产生6,7,所以直接用rand5去实现函数rand7似乎不太好入手。 如果反过来呢?给你rand7,让你实现rand5,这个好实现吗?

一个非常直观的想法就是不断地调用rand7,直到它产生1到5之间的数,然后返回。 代码如下:
\begin{lstlisting}[language=C++]
int Rand5(){
    int x = ~(1<<31); // max int
    while(x > 5)
        x = Rand7();
    return x;
}
\end{lstlisting}
等等,这个函数可以等概率地产生1到5的数吗?首先,它确确实实只会返回1到5这几个数, 其次,对于这些数,都是由Rand7等概率产生的(1/7),没有对任何一个数有偏袒, 直觉告诉我们,Rand5就是等概率地产生1到5的。事实呢?让我们来计算一下, 产生1到5中的数的概率是不是1/5就OK了。比如说,让我们来计算一下Rand5生成1 的概率是多少。上面的函数中有个while循环,只要没生成1到5间的数就会一直执行下去。 因此,我们要的1可能是第一次调用Rand7时产生,也可能是第二次,第三次,…第n次。 第1次就生成1,概率是1/7;第2次生成1,说明第1次没生成1到5间的数而生成了6,7, 所以概率是(2/7)*(1/7),依次类推。生成1的概率计算如下:
\begin{itemize}
\item P(x=1)=1/7\ +\ (2/7)\ *\ 1/7\ +\ (2/7)\^2\ *\ 1/7\ +\ (2/7)\^3\ *\ 1/7\ +\ ...
\item \ \ \ \ \ \ =1/7\ *\ (1\ +\ 2/7\ +\ (2/7)\^2\ +\ ...)\ //\ 等比数列
\item \ \ \ \ \ \ =1/7\ *\ 1\ /\ (1\ -\ 2/7)
\item \ \ \ \ \ \ =1/7\ *\ 7/5
\item \ \ \ \ \ \ =1/5
\end{itemize}
上述计算说明Rand5是等概率地生成1,2,3,4,5的(1/5的概率)。从上面的分析中, 我们可以得到一个一般的结论,如果a > b,那么一定可以用Randa去实现Randb。其中, Randa表示等概率生成1到a的函数,Randb表示等概率生成1到b的函数。代码如下:
\begin{lstlisting}[language=C++]
// a > b
int Randb(){
    int x = ~(1<<31); // max int
    while(x > b)
        x = Randa();
    return x;
}
\end{lstlisting}
回到正题,现在题目要求我们要用Rand5来实现Rand7,只要我们将Rand5 映射到一个能产生更大随机数的Randa,其中a > 7,就可以套用上面的模板了。 这里要注意一点的是,你映射后的Randa一定是要满足等概率生成1到a的。比如,
\begin{itemize}
\item Rand5() + Rand5() - 1
\end{itemize}
上述代码可以生成1到9的数,但它们是等概率生成的吗?不是。生成1只有一种组合: 两个Rand5()都生成1时:(1, 1);而生成2有两种:(1, 2)和(2, 1);生成6更多。 它们的生成是不等概率的。那要怎样找到一个等概率生成数的组合呢?

我们先给出一个组合,再来进行分析。组合如下:
\begin{itemize}
\item 5 * (Rand5() - 1) + Rand5()
\end{itemize}
Rand5产生1到5的数,减1就产生0到4的数,乘以5后可以产生的数是:0,5,10,15,20。 再加上第二个Rand5()产生的1,2,3,4,5。我们可以得到1到25, 而且每个数都只由一种组合得到,即上述代码可以等概率地生成1到25。OK, 到这基本上也就解决了。

套用上面的模板,我们可以得到如下代码:
\begin{lstlisting}[language=C++]
int Rand7(){
    int x = ~(1<<31); // max int
    while(x > 7)
        x = 5 * (Rand5() - 1) + Rand5() // Rand25
    return x;
}
\end{lstlisting}
上面的代码有什么问题呢?可能while循环要进行很多次才能返回。 因为Rand25会产生1到25的数,而只有1到7时才跳出while循环, 生成大部分的数都舍弃掉了。这样的实现明显不好。我们应该让舍弃的数尽量少, 于是我们可以修改while中的判断条件,让x与最接近25且小于25的7的倍数相比。 于是判断条件可改为x > 21,于是x的取值就是1到21。 我们再通过取模运算把它映射到1-7即可。代码如下:
\begin{lstlisting}[language=C++]
int Rand7(){
    int x = ~(1<<31); // max int
    while(x > 21)
        x = 5 * (Rand5() - 1) + Rand5() // Rand25
    return x%7 + 1;
}
\end{lstlisting}
这个实现就比上面的实现要好,并且可以保证等概率生成1到7的数。

让我们把这个问题泛化一下,从特殊到一般。现在我给你两个生成随机数的函数Randa, Randb。Randa和Randb分别产生1到a的随机数和1到b的随机数,a,b不相等 (相等就没必要做转换了)。现在让你用Randa实现Randb。

通过上文分析,我们可以得到步骤如下:
\begin{enumerate}
\item 如果a > b,进入步骤2;否则构造Randa2 = a * (Randa - 1) + Randa, 表示生成1到a2 随机数的函数。如果a2 仍小于b,继教构造 Randa3 = a * (Randa2 - 1) + Randa…直到ak > b,这时我们得到Randak , 我们记为RandA。
\item 步骤1中我们得到了RandA(可能是Randa或Randak ),其中A > b, 我们用下述代码构造Randb:
  \begin{lstlisting}[language=C++]
    // A > b
    int Randb(){
      int x = ~(1<<31); // max int
      while(x > b*(A/b)) // b*(A/b)表示最接近A且小于A的b的倍数
      x = RandA();
      return x%b + 1;
    }
  \end{lstlisting}
\end{enumerate}
从上面一系列的分析可以发现,如果给你两个生成随机数的函数Randa和Randb, 你可以通过以下方式轻松构造Randab,生成1到a*b的随机数。
\begin{itemize}
\item Randab = b * (Randa - 1) + Randb
\item Randab = a * (Randb - 1) + Randa
\end{itemize}
如果再一般化一下,我们还可以把问题变成:给你一个随机生成a到b的函数, 用它去实现一个随机生成c到d的函数。有兴趣的同学可以思考一下,这里不再讨论。
%\lstinputlisting[language=C++]{ch19ten.cpp}


\item[19.11] Design an algorithm to find all pairs of integers within an array which sum to a specified value.

设计一个算法,找到数组中所有和为指定值的整数对。

Solution: 
\begin{description}
\item[时间复杂度O(n)的解法] 我们可以用一个哈希表或数组或bitmap(后两者要求数组中的整数非负)来保存sum-x的值, 这样我们就只需要遍历数组两次即可找到和为指定值的整数对。这种方法需要O(n) 的辅助空间。如果直接用数组或是bitmap来做,辅助空间的大小与数组中的最大整数相关, 常常导致大量空间浪费。比如原数组中有5个数:1亿,2亿,3亿,4亿,5亿。sum为5亿, 那么我们将bitmap中的sum-x位置1,即第4亿位,第3亿位,第2亿位,第1亿位,第0位置1. 而其它位置都浪费了。

如果使用哈希表,虽然不会有大量空间浪费,但要考虑冲突问题。

\item[时间复杂度为O(nlogn)的解法] 我们来考虑一种空间复杂度为O(1),而且实现也很简单的算法。首先,将数组排序。 比如排序后得到的数组a是:-2 -1 0 3 5 6 7 9 13 14。然后使用low和high 两个下标指向数组的首尾元素。如果a[low]+a[high] > sum,那么说明a[high] 和数组中的任何其它一个数的和都一定大于sum(因为它和最小的a[low]相加都大于sum)。 因此,a[high]不会与数组中任何一个数相加得到sum,于是我们可以直接不要它, 即让high向前移动一位。同样的,如果a[low]+a[high] < sum,那么说明a[low] 和数组中的任何其它一个数的和都一定小于sum(因为它和最大的a[high]相加都小于sum)。 因此,我们也可以直接不要它,让low向前移动一位。如果a[low]+a[high]等于sum, 则输出。当low小于high时,不断地重复上面的操作即可。
\end{description}
代码如下:
\lstinputlisting[language=C++]{ch19ele.cpp}

\end{description}
